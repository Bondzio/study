\documentclass[12pt]{article}
\usepackage[english,german]{babel}
\usepackage[utf8]{inputenc}
\usepackage{color}
\usepackage{hyperref}
\usepackage{mathtools}
\setlength{\parindent}{0em} 
\hypersetup{
    colorlinks=true,
    linktoc=all,
    linkcolor=black,
    urlcolor=black
}
%Gummi|065|=)
\title{\Huge\textbf{experimentelle Methoden der Bioinformatik}}
\author{}
\date{}

% set title of table of contents
\renewcommand*\contentsname{Inhalt}

% https://www.sharelatex.com/learn
% http://www.math.ubc.ca/~cautis/tools/latexmath.html
% http://www.golatex.de/wiki/Kategorie:Befehlsreferenz
% https://en.wikibooks.org/wiki/LaTeX/Mathematics

\begin{document}

\begin{titlepage}

\maketitle
\thispagestyle{empty}
\end{titlepage}
\newpage

\begin{titlepage}
\tableofcontents
\thispagestyle{empty}
\end{titlepage}
\newpage

\section{Motifsuche}

\subsection{ChIP-chip und ChIP-seq}

\subsection{crosslinking}

\subsection{sonication}

\subsection{Selektion mittels Antikörpern}

\subsubsection{Eigenschaften von Antikörpern}

\subsection{reverse crosslinking}

\subsection{Chiphybridisierung}

\subsection{Sequencing}

\section{Peak Calling}

\section{CLIP-Seq}

\subsection{ICLIP}

\section{PAR-CLIP}

\section{Protein-Protein-Interaktion}

\section{Tandem Affinity Purification (TAP)}

\subsection{Local clique merging algorithm (LCMA)}

\subsection{Clique Finding Algorithzm (CFA)}

\section{RNA structure probing}

\subsection{chemical probing}

\end{document}