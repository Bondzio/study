\section{Allgemein / Hintergrund}
Messung von Strukturen vs. Messung von Interaktionen\\
Motifsuche:\\
- Proteine (Transkriptionsfaktoren) haben Domaine die Nukleotidsequenzen erkennen\\
- Position weight matrix (PWM), position specific scoring matrix (PSSM)\\
- MEME zum erkennen von Sequenzen / Motifen\\

\section{ChIP-Chip und ChIP-Seq}
ChIP: \textbf{Ch}romatin-\textbf{I}mmuno\textbf{P}recipitation\\
Kein Single Cell Protocol -> es werden Zellpopulationen benötigt\\
Ziel: Man will feststellen an welcher Stelle Proteine binden (\textbf{DNA-Protein-Interaktion\footnote{\url{https://de.wikipedia.org/wiki/Protein-DNA-Interaktion}}})\\
Quellen für Fehler / Ungenauigkeiten: Messung des Populationsmittelwerts\\
ChIP-Chip: Chromatin-Immunoprecipitation Chip\\
ChIP-Seq: Chromatin-Immunoprecipitation DNA-Sequencing

\subsection{Ablauf}
\subsubsection{Crosslinking}
Stabilisierung der Bindungen zwischen DNA und Protein\\
Geschieht reversibel zwischen DNA (\textbf{Chromatin}) und rekombinanten Proteinen
\begin{itemize}
	\item Formaldehyd (CH2O) vernetzt Base (B) mit Proteinen (P-NH2) quer
	\item P-NH2+CH2O $\rightleftharpoons$ PN=CH2+NH2-B $\rightleftharpoons$ PNH-CH2-NH-B
	\item Rekombinant: Biotechnologisch hergestellte Proteine aus genetisch 
veraenderten Organismen
\end{itemize}

\subsubsection{Sonication}
Zerstören und Zerkleinern (fragmentieren) der Zellen, Zellbestandteile und DNA durch Ultraschall\\
(Vorher: Waschen der Zellen mit Protease Inhibitor, Lyse + homogenisieren)
\begin{itemize}
	\item zeitkritisch $\rightarrow$ Länge bestimmt Grad der Zerkleinerung
	\item 200-1000 BP Fragmente im Idealfall
\end{itemize}
Ergebnis sind DNA Fragmente mit gebundenen Proteinen

\subsubsection{Immunoprezipitation (Selektion mittels Antikörper)}
- Antikörper (binden an Beads oder Membranen, Chip/in Gel) binden an rekombinante Proteine oder Protein-TAG (kurze Aminosäuresequenz, markieren Protein)\\
\underline{Aufreinigung:}
\begin{itemize}
	\item Zentrifugation des Prezipitats: Beads+(Protein-DNA) am Boden, Zellfragmente/Rest in Lösung
	\item Abkippen der Lösung
	\item Aufnehmen des Beadspellets in Puffer, erneut zentrifugieren (x-Mal)
	\item Manchmal noch:
	\begin{itemize}
		\item DNase Verdau der DNA in Lösung
		\item Aufheben der DNA in Lösung, als total-Chromatin-Probe
	\end{itemize}
\end{itemize}

\subsubsection{Reverse Immunoprezipitation}
Durch Aufreinigungsschritte sind Beads/Gel/Chip idealerweise frei von Zellfragmenten/ungebundener DNA.\\Umkehren der IP mit Elutionspuffer$\rightarrow$ Antikörper von DNA+Proteine trennen\\
$\rightarrow$Salzgehalt und PH-Wert an Rückreaktion angepasst

\subsubsection{Reverse Cross Linking}
\begin{itemize}
	\item Thermische Zerstörung der Bindung zw. Protein und DNA
	\item Salzgehalt des Buffer angepasst auf Rückreaktion
	\item Proteinase K und RNase bauen Proteine und RNA ab(zur Aufreinigung)
	\item Extraktion der übrig gebliebenen DNA durch Zentrifuge
\end{itemize}

\subsubsection{Auswertung}
\begin{itemize}
	\item \textbf{Chiphybridisierung:}
		\begin{itemize}
			\item Hybridisierung der DNA an Microarray
			\item Färbung der DNA
			\item Messung der Farbintensität
			\item \underline{mit dem ChIP Background kann ich nichts anfangen...}
		\end{itemize}
	\item \textbf{Sequencing:}
		\begin{itemize}
			\item Hochdurchsatzsequenzierung der aufgereinigten DNA
			\item DNA extrahieren$\rightarrow$DNA fragmentieren$\rightarrow$Primer an Fragmente$\rightarrow$Sequenzierung
			\item Herausrechnen der Primer (idealerweise kennt man sie)
			\item Quality control$\rightarrow$Phred-score Berechnung (Güte der erkannten Nukleobase)$\rightarrow$Cutoff bei zu niedrigem Phred-score
			\item Mapping des sequenzierten Teilstücks auf Genom
		\end{itemize}
\end{itemize}

\subsection{Probleme/Fehler}
\textbf{Cross-Linking}
\begin{itemize}
	\item \textbf{FN:} Protein an DNA gebunden, aber kein Cross-Linking
	\item \textbf{FP:} Proteine, die sehr nahe an der DNA sind, aber ungebunden, werden auch cross linked
\end{itemize}

\textbf{Sonication}
\begin{itemize}
	\item Größe der Fragmente abhängig von Ultraschalleinsatz - zeitkritisch!
	\item Kürzere und längere Fragmente können Informationen enthalten
\end{itemize}

\textbf{Immunoprecipitation}
\begin{itemize}
	\item \textbf{FP:} Mangelnde Reinheit der rekombinanten Proteine; Spezifität der heterophilen Antikörper zu gering
	\item Aufreinigung führt zu \textbf{FP} und \textbf{FN}
\end{itemize}

\textbf{Chip:} \textbf{FN:} Hybridisierung nicht effektiv genug

\subsection{Antikörper}
\begin{itemize}
	\item Antikörper bindet spezifisch und sensitiv
	\item Antikörper sind fixiert an:
	\begin{itemize}
		\item Beads
		\item Chip (kein Microarray)
		\item Gel
	\end{itemize}
	\item Antikörper werden im Experiment erzeugt
\end{itemize}

\begin{tabular}{cc}
  \textbf{polyclonal} & \textbf{monoclonal}\\
  Aufbrechen der Proteine in kurze\\ 
  Aminosäureketten (Peptide) & Aufbrechen der Proteine in Peptide\\
  $\Downarrow$ & $\Downarrow$\\
  Peptide in Ratte/Maus geimpft & Peptide in Ratte/Maus geimpft\\
  $\Downarrow$ & $\Downarrow$\\
  extrahieren der B-Lymphozyten aus Serum & extrahieren der B-Lymphozyten aus Milz\\
  $\Downarrow$ & $\Downarrow$\\
  Extrahieren der Antikörper & Fusionierung der B-Lymphozyten mit\\ 
   aus den Lymphozyten & Plasmazellen aus Myelom\\
   & (Krebszelle - 'unsterblich')\\
  $\Downarrow$ & $\Downarrow$\\
  & Hybrid erzeugt (unsterblich + Antikörper)\\
  $\Downarrow$ & $\Downarrow$\\
  & Testen der Hybride auf Antigene\\
  $\Downarrow$ & $\Downarrow$\\
  ernten verschiedenen Antikörper & ernten identischer Antikörper\\
  $\Downarrow$ & $\Downarrow$\\
  billig & teuer
\end{tabular}