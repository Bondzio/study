\section{Allgemein / Hintergrund}
Messung von Strukturen vs. Messung von Interaktionen\\
Motifsuche:\\
- Proteine (Transkriptionsfaktoren) haben Domaine die Nukleotidsequenzen erkennen\\
- Position weight matrix (PWM), position specific scoring matrix (PSSM)\\
- MEME zum erkennen von Sequenzen / Motifen\\

\section{ChIP-Chip und ChIP-Seq}
ChIP: \textbf{Ch}romatin-\textbf{I}mmuno\textbf{P}recipitation\\
Kein Single Cell Protocol -> es werden Zellpopulationen benötigt\\
Ziel: Man will feststellen an welcher Stelle Proteine binden\\
Quellen für Fehler / Ungenauigkeiten: Messung des Populationsmittelwerts\\
ChIP-Chip: Chromatin-Immunoprecipitation Chip\\
ChIP-Seq: Chromatin-Immunoprecipitation DNA-Sequencing

\subsection{Ablauf}
\subsubsection{Crosslinking}
Stabilisierung der Bindungen zwischen DNA und Protein\\
Geschieht reversibel zwischen DNA (\textbf{Chromatin}) und rekombinanten Proteinen\\
\hspace*{10mm}- Formaldehyd (CH2O) vernetzt Base (B) mit Proteinen (P-NH2) quer\\
\hspace*{10mm}- P-NH2+CH2O $\rightleftharpoons$ PN=CH2+NH2-B $\rightleftharpoons$ PNH-CH2-NH-B\\
\hspace*{10mm}- Rekombinant: Biotechnologisch hergestellte Proteine aus genetisch\\
\hspace*{10mm}veraenderten Organismen\\

\subsubsection{Sonication}
Zerstören und Zerkleinern (fragmentieren) der Zellen, Zellbestandteile und DNA durch Ultraschall\\
(Vorher: Waschen der Zellen mit Protease Inhibitor, Lyse + homogenisieren)\\
\hspace*{10mm}- zeitkritisch $\rightarrow$ Länge bestimmt Grad der Zerkleinerung\\
\hspace*{10mm}- 200-1000 BP Fragmente im Idealfall\\
Ergebnis sind DNA Fragmente mit gebundenen Proteinen\\

\subsubsection{Immunoprezipitation (Selektion mittels Antikörper)}
\hspace*{10mm}- Antikörper (binden an Beads oder Membranen, Chip/in Gel) binden an rekombinante Proteine\\
\hspace*{13mm}oder Protein-TAG (kurze Aminosäuresequenz, markieren Protein)\\
\hspace*{10mm}-	Aufreinigung:\\
\hspace*{15mm}$\rightarrow$ Zentrifugation des Prezipitats: Beads+(Protein-DNA) am Boden,\\
\hspace*{21mm}Zellfragmente/Rest in Lösung\\
\hspace*{15mm}$\rightarrow$ Abkippen der Lösung\\
\hspace*{15mm}$\rightarrow$ Aufnehmen des Beadspellets in Puffer, erneut zentrifugieren (x-Mal)\\ 
\hspace*{15mm}Manchmal noch\\
\hspace*{15mm}$\rightarrow$ DNase Verdau der DNA in Lösung\\
\hspace*{15mm}$\rightarrow$ Aufheben der DNA in Lösung, als total-Chromatin-Probe

\subsubsection{Reverse Immunoprezipitation}
Durch Aufreinigungsschritte sind Beads/Gel/Chip idealerweise frei von Zellfragmenten/ungebundener DNA.\\Umkehren der IP mit Elutionspuffer$\rightarrow$ Antikörper von DNA+Proteine trennen\\
$\rightarrow$Salzgehalt und PH-Wert an Rückreaktion angepasst

\subsubsection{Reverse Cross Linking}
\hspace*{10mm}-	Thermische Zerstörung der Bindung zw. Protein und DNA\\
\hspace*{10mm}- Salzgehalt des Buffer angepasst auf Rückreaktion
\hspace*{10mm}-	Proteinase K und RNase bauen Proteine und RNA ab(zur Aufreinigung)\\
\hspace*{10mm}-	Extraktion der übrig gebliebenen DNA durch Zentrifuge\\

\subsubsection{Auswertung}
\hspace*{10mm}\textbf{Chiphybridisierung}\\
\hspace*{20mm}- Hybridisierung der DNA an Microarray\\
\hspace*{20mm}- Färbung der DNA\\
\hspace*{20mm}- Messung der Farbintensität\\
$\rightarrow$\textit{\textbf{mit dem ChIP Background kann ich nichts anfangen...}}$\leftarrow$\\
\hspace*{10mm}\textbf{Sequencing}\\
\hspace*{20mm}Hochdurchsatzsequenzierung der aufgereinigten DNA.\\
\hspace*{20mm}$\rightarrow$DNA extrahieren$\rightarrow$DNA fragmentieren$\rightarrow$Primer an Fragmente$\rightarrow$Sequenzierung\\
\hspace*{20mm}$\rightarrow$Herausrechnen der Primer (idealerweise kennt man sie)$\rightarrow$\\
\hspace*{20mm}Quality control$\rightarrow$Phred-score Berechnung (Güte der erkannten\\
\hspace*{20mm}Nukleobase)$\rightarrow$Cutoff bei zu niedrigem Phred-score$\rightarrow$Mapping des\\
\hspace*{20mm}sequenzierten Teilstücks auf Genom

\subsection{Probleme/Fehler}
\textbf{Cross-Linking}\\
\hspace*{10mm}\textbf{FN:} Protein an DNA gebunden, aber kein Cross-Linking\\
\hspace*{10mm}\textbf{FP:} Proteine, die sehr nahe an der DNA sind, aber ungebunden, werden\\ \hspace*{19mm}auch cross linked\\
\\
\textbf{Sonication}\\
\hspace*{10mm}- Größe der Fragmente abhängig von Ultraschalleinsatz – zeitkritisch!\\
\hspace*{10mm}- Kürzere und längere Fragmente können Informationen enthalten\\
\\
\textbf{Immunoprecipitation}\\
\hspace*{10mm}\textbf{FP:} Mangelnde Reinheit der rekombinanten Proteine; Spezifität der\\
\hspace*{19mm}heterophilen Antikörper zu gering\\
\hspace*{19mm}Aufreinigung führt zu \textbf{FP} und \textbf{FN}\\
\\
\textbf{Chip}\\
\hspace*{10mm}\textbf{FN:} Hybridisierung nicht effektiv genug

\subsection{Antikörper}
- Antikörper bindet spezifisch und sensitiv \\
- Antikörper sind fixiert an:\\
\hspace*{15mm} - Beads\\
\hspace*{15mm} - Chip (kein Microarray)\\
\hspace*{15mm} - Gel\\
- Antikörper werden im Experiment erzeugt\\
\\
\begin{tabular}{cc}
  \textbf{polyclonal} & \textbf{monoclonal}\\
  Aufbrechen der Proteine in kurze\\ 
  Aminosäureketten (Peptide) & Aufbrechen der Proteine in Peptide\\
  $\Downarrow$ & $\Downarrow$\\
  Peptide in Ratte/Maus geimpft & Peptide in Ratte/Maus geimpft\\
  $\Downarrow$ & $\Downarrow$\\
  extrahieren der B-Lymphozyten aus Serum & extrahieren der B-Lymphozyten aus Milz\\
  $\Downarrow$ & $\Downarrow$\\
  Extrahieren der Antikörper &Fusionierung der B-Lymphozyten mit\\ 
   aus den Lymphozyten & Plasmazellen aus Myelom\\
   & (Krebszelle - 'unsterblich')\\
  $\Downarrow$ & $\Downarrow$\\
  Antikörper & Hybrid erzeugt (unsterblich + Antikörper)\\
  & $\Downarrow$\\
  & Testen der Hybride auf Antigene\\
  & $\Downarrow$\\
  & ernten spezifischer Antikörper
\end{tabular}