%Vorlesung 3- 11.04.2016
\subsubsection{Backtracking von Zuker}
Stapel unaufgekl\"arter Strukturen: $i$, $j$ und Typ
\begin{itemize}
	\item[] \textbf{push} $F(1,n)$ on stack
	\item[] \textbf{Wiederhole, solange Stack nicht leer ist:}
	\begin{itemize}
		\item[] \textbf{pop}
		\item[] \textit{if} $F(i,j)=F(i+1, j)$:
			\begin{itemize}
				\item[] $\rightarrow$ \textbf{push} $F(i+1, j)$ on stack
			\end{itemize}
		\item[] $else$:
			\begin{itemize}
				\item[] Suche $k$, mit $i < k \le j$, sodass $F(i,j) = C(i,k) + F(k+1,j)$:
				\item[] $\rightarrow$ \textbf{push} $C(i, k)$ and $F(k+1, j)$ on stack
			\end{itemize}
		\item[]
		\item[] $if~C(i,j) = \mathcal{H}(i,j)$:
			\begin{itemize}
				\item[] $\rightarrow i,j$ als Basenpaar markieren
			\end{itemize}
		\item[] $else~if~\exists~k,l$, mit $i < k < l < j$, sodass $C(i,j) = \mathcal{I}(i,j,k, l) + C(k,l)$:
			\begin{itemize}
				\item[] $\rightarrow i,j$ als Basenpaar markieren
				\item[] $\rightarrow$ \textbf{push} $C(k,l)$ on stack
			\end{itemize}
		\item[] $else$:
			\begin{itemize}
				\item[] Suche $u$, mit $i < u < j$, sodass $C(i,j) = a + M(i+1, u) + M^1(u+1, j-1)$
				\item[] $\rightarrow i,j$ als Basenpaar markieren
				\item[] $\rightarrow$ \textbf{push} $M(i+1, u)$ and $M^1(u+1,j-1)$ on stack
			\end{itemize}
		\item[]
		\item[] $if~M(i,j) = M(i,j-1) + c$:
			\begin{itemize}
				\item[] $\rightarrow$ \textbf{push} $M(i, j-1)$ on stack
			\end{itemize}
		\item[] $else~if~\exists~k$, mit $i < k < j$, sodass $(k-i) \cdot c + b + C(k,j)$:
			\begin{itemize}
				\item[] $\rightarrow$ \textbf{push} $C(k,j)$ on stack
			\end{itemize}
		\item[] $else$:
			\begin{itemize}
				\item[] Suche $k$, mit $i < k < j$, sodass $M(i,j) = M(i, k-1) + b + C(k,j)$:
				\item[] $\rightarrow$ \textbf{push} $M(i, k-1)$ and $C(k,j)$ on stack
			\end{itemize}
		\item[]
		\item[] $if M^1(i,j) = M^1(i,j-1) + c$:
			\begin{itemize}
				\item[] $\rightarrow$ \textbf{push} $M^1(i, j-1)$ on stack
			\end{itemize}
		\item[] $else$:
			\begin{itemize}
				\item[] $\rightarrow$ \textbf{push} $C(i, j)$ on stack
			\end{itemize}
	\end{itemize}
\end{itemize}


\subsection{Suboptimale Strukturen}

\subsubsection{ Zuker-Suboptimals (\textit{1989})}
Berechnet suboptimale Strukturen, indem für jedes \textit{m\"ogliche} Basenpaar die beste Struktur, insgesamt quadratisch viele Strukturen, berechnet wird. Dies wir durch ein Verdoppeln der Input-Sequenz erreicht. Die beiden Sequenzen werden dann wie folgt aneinander geh\"angt:
\begin{itemize}
\item[]$1, .. .,n, n+1, ..., 2n$
\item[]Optimale Energie f\"ur $BP(i,j)=C(i,j)+C(j, n+i)$
\end{itemize}
Wird der Zuker-Algorithmus auf diese verdoppelte Sequenz durchgef\"uhrt erh\"alt man Ergebnisse...\textbf{ Wer kann's erkl\"aren?}

\subsubsection{Wuchty-Algorithmus}
\begin{enumerate}
\item\textbf{Forward-Algorithmus} kann Nussinov- oder Zuker-Algorithmus sein
\item\textbf{Backtracking}:\\
Stapel enth\"alt mehr Information als bei Zuker:  i, j, Typ, Stuktur und $\Delta\varepsilon$ (Energie, die noch verf\"ugbar ist)
\end{enumerate}

\paragraph{Wuchty-Backtrack von Nussinov-Matrizen}
Beim Wuchty-Backtracking von Nussinov-Matrizen werden, im Gegensatz zum ``normalen''
Backtrack, \emph{Refinements} aufgehoben und weiterverfolgt, die eine h\"ohere Anzahl von Basenpaaren 
haben als vorher \"{u}ber einen Schwellenwert als Minimum festgelegt. 
\begin{equation} \label{eq:wc}
  P_{\mathcal{S}'} \geqslant P_{max}-\Delta_{Schwellenwert}
\end{equation}
Erf\"ullt nun das Refinement $\mathcal{S}$ mit seiner maximalen Anzahl Basenpaare $P$ \eqref{eq:wp} das 
Kriterium f\"ur gew\"unschte Suboptimale Strukturen \eqref{eq:wc}, wird $\mathcal{S}'$ f\"ur weitere 
\emph{Refinements} auf den \"ubergeordneten Stack $R$ geschoben. 
\begin{equation} \label{eq:wp}
	P_{\mathcal{S}'} = |\mathcal{P}| + 1 + P_{i+1,k-1} + P_{k+1,j} + \sum_{[a,b] \in \sigma} P_{a,b}
\end{equation}
\begin{description}
	\item[Refinement] Beim optimalen Nussinov Backtracking wird \emph{ein} Tupel 
		($\mathcal{S}$, bedeutet \emph{partielle Struktur}) aus einem Stack mit 
		unaufgekl\"arten Sequenzintervallen ($\sigma$) und eine Menge von 
		Basenpaaren ($\mathcal{P}$), $\mathcal{S}=(\sigma, \mathcal{P})$ bearbeitet. 
		Am Ende des Backtrackings ist $\sigma$ leer und $\mathcal{P}$ m\"oglichst groß. 
		\emph{Refinement} bezeichnet nun den Schritt, ein Sequenzintervall $[i,j]$ vom 
		Stack $\sigma$ zu nehmen, in ein kleineres ($[i+1,j]$), bzw. zwei kleinere Intervalle 
		($[i+1,k-1], [k+1,j]$) aufzuteilen, diese jeweils wieder auf den Stack zu 
		schieben, $\mathcal{P}$ gegebenenfalls um ein Basenpaar zu erweitern 
		($\mathcal{P} \cup \{i \cdot k\}$) und damit zusammengefasst 
		$\mathcal{S}$ zu $\mathcal{S}'$ zu ``refinen''.
\end{description}

\paragraph{Wuchty-Backtrack von MFE-Matrizen}
Die partielle Struktur $\mathcal{S}$ wird hier mit der totalen 
freien Energie aller Loops $(E_{L\mathcal{S}})$ aus der partiellen Struktur erweitert 
zu $\mathcal{S}=(\sigma,\mathcal{P}, E_{L\mathcal{S}})$.
Außerdem wird der Stack $\sigma$ erweitert um den Typ der Matrix ($F, C, M, M^1$), 
in der weitergearbeitet werden soll (z.B. $\sigma = \{[i,j]_{M}\}$). 
\begin{equation} \label{eq:wmc}
	E_{\mathcal{S}'} \leqslant E_{min} + \delta
\end{equation}
Wenn zum Beispiel (Zuker, Fall 1)
$F_{i+1,j} + E_{L\mathcal{S}} + \sum_{[k,l] \in \sigma} E_{[k,l]} \leqslant E_{min} + \delta$ 
gilt, also das vorher gew\"ahlte Kriterium f\"ur suboptimale Faltungen \eqref{eq:wmc} 
erf\"ullt ist, 
wird $\mathcal{S}' = (\{[i+1,j]_F\} \cup \sigma; \mathcal{P}; E_{L\mathcal{S}})$ auf den 
Stack $R$ geschoben.

\paragraph{Thermodynamik von Molek\"ulen}
\begin{itemize}
\item[]viele Zust\"ande m\"oglich
\item[]Die Boltzmann-Statistik besagt, dass die Wahrscheinlichkeit $p$ einen Zustand der Energie $E$ mit einem Teilchen besetzt zu finden, proportional ist zum Boltzmann-Faktor:
	\begin{itemize}
	\item[$\circ$]$Bolzmannfaktor=e^{-\frac{E}{k_B\*T}} $, mit Energie $E$, Boltzmannkonstante $k_B$ und der absoluten Temperatur $T$
	\item[$\circ$]$p(E)\propto e^{-\frac{E}{k_B\*T}}$, $\beta=\frac{1}{k*T}$
	\item[$\circ$]f\"ur eine Struktur $S$ und eine Energie $E(S)$ gilt:
	\item[$\circ$]$p(S)\propto e^{-\beta*E(S)}$
	\item[$\circ$]$\sum p(S) = 1 \rightarrow$ Summe aller Wahrscheinlichkeiten aller Strukturen
	\item[$\circ$]\begin{equation}
	p(S)= \frac{e^{-\beta*E(S)}}{\sum \limits_{s'=s1}^{sn} e^{-\beta*E(S)}}
	\end{equation}wobei \begin{equation}
	\sum \limits_{s'=s1}^{sn} e^{-\beta*E(S)}=Z
	\end{equation}Z=\textbf{Zustandssumme} (partition function)
	\item[$\circ$] D.h., dass eine Wahrscheinlichkeit f\"ur jede Eigenschaft $x$ (z.B. spezifisches Basenpaar) meines Molek\"uls wie folgt berechnet werden kann: \begin{equation}
	p(x)=\frac{\sum \limits_{x}  e^{-\beta*E(x)}}{Z}
	\end{equation}
	wobei die Summierung über alle Strukturen l\"auft, die diese Eigenschaft $x$ besitzen
	\end{itemize}
\end{itemize}




