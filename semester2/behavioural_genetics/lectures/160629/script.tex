\section{Vorlesung 29.06.2016}

\subsection{Erwartungen}
 - was kann die Umwelt über die Zukunft aussagen?
\\\\
\underline{aversiver Lernversuch:}
\begin{itemize}
	\item mit Bestrafung (aversiv) statt Belohnung (appetitiv)
	\item Training mit Geruch ohne Konsequenzen und mit Geruch und anschließender Bestrafung
	\item $\Rightarrow$ Geruch mit anschließender Bestrafung wird gemieden
\end{itemize}

\underline{umgekehrter Lernversuch:}
\begin{itemize}
	\item erst Bestrafung, dann Geruch
	\item $\Rightarrow$ Geruch wird bevorzugt bzw. als Belohnung wahrgenommen, da nach einer Bestrafung eine Belohnung folgt
\end{itemize}

\underline{Blocken von Dopamin:}
\begin{itemize}
	\item durch \textcolor{red}{???Transgenically???} bei Dopamin-produzierenden Nervenzellen
	\item $\Rightarrow$ verschlechtert aversives Lernen
	\item $\Rightarrow$ verschlechtert \underline{nicht} appetitives Lernen
\end{itemize}

\textbf{relief learning} $\rightarrow$ erlösendes Lernen
\\\\
\underline{Merkmale von :}
\begin{itemize}
	\item braucht wiederholendes Training
	\item funktioniert mit verschiedenen Gerüchen
	\item benötigt eine mittelere Shock Intensität
	\item ist stabil für 4 bis 24 Stunden
	\item \textbf{funktioniert auch beim Menschen}
\end{itemize}

\subsection{angeborenes Verhalten}
 - Beispiele: Sexualverhalten, Aggression
\\\\
\textbf{Zweck:} Systemvereinfachung $\Rightarrow$ muss nicht erst erlernt werden
\\\\
\underline{wissenschaftlich Relevant/Interessant weil:}
\begin{itemize}
	\item relativ leichte Induzierbarkeit im Labor
	\item Meßbarkeit
	\item Invertebratenmodelle
\end{itemize}

\underline{Balz von Drosophila:} dient der Fortpflanzung aber auch Artbildung, Pherhormonfunktion

\subsubsection{genetische Geschlechtsbestimmung bei Drosophila}
 - entscheidend ist Zahlenverhältnis der X-Chromosomen zu den Autosomen\footnote{Als Autosomen werden in der Genetik jene Chromosomen bezeichnet, die nicht zu den Gonosomen (Geschlechtschromosomen) gehören.}
 - 1X/1A $\rightarrow$ männlich, 1X+n/1A $\rightarrow$ weiblich
\\\\
\textbf{Vorgang der Gechlechtsdetermination:}
\begin{itemize}
	\item Sxl-Gen fungiert als Schaltergen
	\item AN: Tra aktiv $\rightarrow$ weiblich
	\item AUS: Tra inaktiv $\rightarrow$ männlich
\end{itemize}

Tra: Transkriptionsfaktor; bindet an Fru-Gen $\rightarrow$ entsprechende Splice-Variante wird ausgebildet

\subsubsection{somatische Geschlecht und sexuelle Verhalten}
\textbf{Frage:} Ist das somatische Geschlecht und das sexuelle Verhalten trennbar?
\begin{itemize}
	\item bei Expression des fruitless-Gens in weiblichen Drosophilas lässt sich männliches Balzverhalten beobachten
	\item zwei Splicevarianten des fruitless-Gens für männlich (Fru$^M$) und weiblich (Fru$^F$):
	\begin{itemize}
		\item Fru$^M$ in Weibchen führt zu selektivem anbalzen von WT-Weibchen
		\item Fru$^F$ in Männchen führt zu fehlen des männlichen Balzverhaltens
	\end{itemize}
	\item kein lebensnotwendiges Sexualverhalten, wenn entsprechendes Gen komplett fehlt
\end{itemize}

\underline{Umkehrung der Rollen:} Weibchen mit Fru$^M$ balzen Männchen an, die weibliche Pheromone produzieren

\subsubsection{Sexualdimorphismus im Gehirn}
Der Sexualdimorphismus im Gehirn von Drosophila ensteht durch den Zelltod der überzähligen Neuronen bei Weibchen. Fru$^M$ verhindert diesen Zelltod. Fru$^F$ fördert Zelltod.
\\\\
Fru$^M$ wirkt auch im adulten Gehirn.
\\\\
\textbf{Frage:} Mastergen für angeborenes Verhalten?
\begin{itemize}
	\item Spleißprodukt Fru$^M$ ist Mastergen für männliches Sexualverhalten
	\item Expression ist neuronal, 2\% aller Neuronen, sowohl in PNS (peripheren Nervensystem) als auch im ZNS (zentralen Nervensystem)
	\item in Weibchen werden überzählige Neuronen durch Abwesenheit von Fru$^M$ durch Zelltod eliminiert
\end{itemize}

$\Rightarrow$ bei Drosophila ist Sexualverhalten ist durch genetische Programmierung vom somatischen Geschlecht trennbar\\
$\Rightarrow$ beim Menschen ist genetische Prädisposition für die Trennung von sexueller Orientierung und somatischem Geschlecht wahrscheinlich, jedoch nicht allein bestimmend (siehe Statistiken)