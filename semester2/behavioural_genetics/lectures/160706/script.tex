\section{Vorlesung 06.07.2016}

\subsection{Aggression}

\textbf{Wo spielt Aggression eine Rolle? Welche Funktion hat sie?}\\
\begin{itemize}
	\item Interspezifisch (Aggression zwischen Arten)
	\begin{itemize}
		\item im Dienst der Ernährung oder des Überlebens
		\begin{itemize}
			\item Angriff
			\item Verteidigung
		\end{itemize}
	\end{itemize}
	\item Intraspezifisch (Aggression zwischen Artgenossen)
	\begin{itemize}
		\item Spielverhalten
		\item Sexualverhalten
		\item Revierverhalten
		\item Sozialverhalten
		\item Frustrationsverhalten
		\item Kollektiver Angriff auf Gruppenfeind
	\end{itemize}
\end{itemize}

\textbf{soziale Insekten}
\begin{itemize}
	\item Disziplinierung der Arbeiterinnen beim Kastenwechsel
	\item Erkennung von Selbst und Nicht-Selbst ("Nestgeruch" über Kohlenwasserstoffprofile auf der Kutikula)
	\item Koordinierte Überfälle von Ameisen auf unterlegene Nester
	\item Drohnenmord
	\item Königinnenmord
	\item Durchgängig sind Amine als Transmitter, Peptide und Steroidhormone als Neuromodulatoren und Neurohormone im Spiel, z. B. Serotonin, Dopamin, Octopamin etc.
\end{itemize}

\textbf{nicht-soziale Insekten}
\begin{itemize}
	\item Hier wurde vor allem mit Grillen gearbeitet, auf deren Kämpfe z. B. in China gewettet wird.
	\item (Hier scheinen Octopamin und Dopamin nicht notwendig für innerartliches Aggressionsverhalten zu sein)
	\item In Verlierern und Gewinnern kann die Injektion von Drogen zu entgegengesetzen Ergebnissen führen
	\item Werden Verlierer gezwungen zu fliegen, restauriert sich ihre Kampfbereitschaft (das wird bei menschlichen Wettern bei Grillenkämpfen als Trick gern eingesetzt)
\end{itemize}

\textbf{Krebse}
\begin{enumerate}
	\item Das Studium des Nervensystems der Krebse ist sehr alt (TH Huxley, S. Freud und G. Retzius untersuchten bereits dessen Anatomie)
	\item Flußkrebs und Hummer sind ideal, um die neuronale Basis des Aggressionsverhaltens zu studieren, weil
\begin{enumerate}
		\item Relativ wenige, sehr große, gut charakterisierte aminerge Neurone existieren
		\item Die Schaltkreise fiir Aggressionsverhalten bekannt sind
		\item Die Aminkonzentrationen leicht gemessen werden können
		\item Das Aggressionsverhalten quantifizierbar ist
		\item Krebse als Individualisten leben und kämpfen, es gibt keine Koalitionen und keine Komplikationen durch die Erkennung von Verwandtschaft
	\end{enumerate}
\end{enumerate}

\subsubsection{Aggressionsverhalten von Drosophila}
\begin{enumerate}
	\item Invertebraten sind für das Studiums des Aggressionsverhaltens geeignet
	\item Drosophila hat als genetisches Modellsystem zunehmen auch auf diesem Gebiet gewonnen
	\item das Aggressionsverhalten verändert sich mit der Erfahrung (Lernverhalten)
	\item im Aggressionsverhalten gibt es einen Sexualdimorphismus
	\item Männchen etablieren eine Dominanzhierarchie (Territorialverhalten), Weibchen nicht
	\item das fruitless-Gen bestimmt nicht nur den Sexualpartner, sondern auch die Art zu kämpfen
\end{enumerate}