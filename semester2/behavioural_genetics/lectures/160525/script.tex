\section{Vorlesung 25.05.2016}

\subsection{Struktur -- Funktion: Axonales Wachstum}

\begin{itemize}
	\item Wachstumskegel bilden ständig sehr bewegliche Filopodien (bis 50 $\mu$m), die auch wieder zurückgebildet werden können
	\item die Beweglichkeit beruht auf der Polymerisation und Depolymerisation von Actin-Mikrofilamenten, was mit einem Membran-Turn-over (Endo-Exocytose von Membranvesikeln) verbunden ist
	\item Manipulation: Dynamin-Gen, Störung der Endocytose, ts-Allel (shibire-mutation $shi^{ts}$)
\end{itemize}

\textbf{Chemoaffinitätstheorie:} Erkennung der Zielzellen durch chem. Markierung\\\\
4 Mechanismen der Wegfindung:
\begin{itemize}
	\item Klassische Chemotaxis = gerichtetes Wachstum nach Stoffgradienten
	\begin{enumerate}
		\item anziehend
		\item abstoßend
	\end{enumerate}
	\item Kontaktführung (contact guidance) = Präferenz für spezielle Substrate (selektive Ahhäsion)
	\begin{enumerate}[resume]
		\item anziehend (Polyornithrin)
		\item abstoßend (Palladium)
	\end{enumerate}
\end{itemize}

\textbf{Wachstumsgeschwindigkeit:} bis 1mm / Tag

\subsubsection{Extrazelluläre Matrixmoleküle (Substrat-Adhäsionsmoleküle - SAM)}
 - Wachstumsunterlagen in Zellkulturen
\\\\
Beispiele:
\begin{itemize}
	\item Fibronectin: 
	\begin{itemize}
		\item Dimer aus nicht identischen, homologen Ketten
		\item Dimerbildung erhöht Variabilität
		\item Bindungstellen u.a. für Fibrin, Heparin, Collagen
	\end{itemize}
	\item Laminin
	\begin{itemize}
		\item Trimer aus umeinandergewundenen, nicht identischen, homologen Ketten
		\item Bindungstellen u.a. für Heparin, Collagen I + IV, spezif. Lamininrezeptoren auf Zelloberflächen
	\end{itemize}
\end{itemize}

\textbf{SAM-Rezeptoren in der Zellmembran -- Integrine}
\begin{itemize}
	\item Glycoproteine
	\item Ligandensequenz: Arg - Gly - Asp (u.a. für Fibronectin)
	\item Zwei Untereinheiten, $\alpha + \beta$; $\alpha$ – sehr variabel, verantwortlich für die Spezifität der Bindung
	\item Positionsspezifische Expression
\end{itemize}

Molekülfamilien:
\begin{itemize}
	\item Neurotrope Faktoren (NTFs)
	\item Netrine
	\item Semaphorine
\end{itemize}

\textbf{Ein Wachstumsfaktoren kann in Abhängigkeit vom Rezeptor, auf den er trifft, unterschiedliche, ja entgegengesetzte Wirkungen haben.}\\\\
Beispiel: NGF $\rightarrow$ Zelltod/Überleben

\subsubsection{Zelladhäsionsmoleküle - CAMs}

\begin{itemize}
	\item Angehörige der Immunoglobulin-Superfamilie ($Ca2$-unabhängige CAMs)
	\begin{itemize}
		\item neuronale CAM der Vertebraten (NCAM)
		\item Down Syndrom Zelladhäsionsmolekül (DSCAM)\footnote{\url{https://en.wikipedia.org/wiki/DSCAM}}: Zielfindung gestört
		\item Roundabout-Protein (Robo)
		\item \textbf{Zelladhäsionsfunktion wahrscheinlich älter als Funktion bei Immunabwehr}
	\end{itemize}
	\item $Ca^2$-abhängige CAMs
		\begin{itemize}
			\item Cadherine: Die Aktivität des Rezeptors nimmt Einfluß
auf die Stabilität der Cadherin-Struktur der Synapse
			\item Protocadherine: Die Genstruktur der Protocadherine erinnert mit den konstanten und variablen Sequenzen an die Struktur der Immunglobulingene (Rekombination zum expremierbaren Gen)
		\end{itemize}
\end{itemize}

\subsubsection{Variablität von Zelladhäsionsmolekülen}
\begin{itemize}
	\item Unterschiedliche zeitl./räumliche Expressionsmuster
	\item Posttranslationale Modifikation (z.B. Glycolysierung, Polysialsäreketten)
	\item Dimere / Multimere
	\item Ggfs. Ca-abhängige Modulation
	\item RNA-Editing
	\item Alternatives Splicen
	\begin{itemize}
		\item NCAM: 192 verschiedene Proteinvarianten, Maus-Gen 80 kb)
		\item Dscam: 95 alternativ gespeißte Exons (in Clusteranordnung) theoretisch 38.016 verschiedene mRNAs, bisher einige Hundert verifiziert
		\item SynCAM: prä- und postsynaptisch, interagieren über PDZ-Domäne, induziert Synapsenbildung
	\end{itemize}
\end{itemize}

\textbf{Ligand-Rezeptorbeziehungen:}\\\\
\begin{tabular}{lcr}
	\textbf{Liganden} & \textbf{Rezeptoren} & \textbf{Reaktion des Wachstumkegels}\\
	Netrin & UNC-40/DCC & Anziehung (Abstoßung)\\
	Netrin & UNC-5 & Abstoßung\\
	Slit & SAX-3/ROBO & Anziehung (Abstoßung)\\
	Semaphorin & Neuropilin/Plexin & Anziehung (Abstoßung)\\
	Ephrin & EPH Receptors & Anziehung (Abstoßung)\\
\end{tabular}

Myelinproduzierende Oligodendrocyten sind eine schlechtes Substrat für
Zellwachstum:
\begin{itemize}
	\item verirrte Axone können sich nicht bereits vorhandenen Faserstraktern anschließen (Kontaktinhibition)
	\item Verlust der Regenerationsfähigkeit im ZNS der höheren Vertebraten (im PNS und niederen Vertebraten keine Inhibition gute
Regenerationsfähigkeit)
	\item Nogo, Nogo-Rezeptor, p75
\end{itemize}