\documentclass[12pt,a4paper]{article}
\usepackage[english,german]{babel}
\usepackage[utf8]{inputenc}
\usepackage{color}
\usepackage{hyperref}
\usepackage{mathtools}
\usepackage{amsmath}
\usepackage{graphicx}
\usepackage{enumitem}

\usepackage{geometry}
\geometry{
  left=3cm,
  right=3cm,
  top=3cm,
  bottom=4cm,
  bindingoffset=5mm
}

\setlength{\parindent}{0em} 
\hypersetup{
    colorlinks=true,
    linktoc=all,
    linkcolor=black,
    urlcolor=black
}

% Hurenkinder und Schusterjungenregel
\clubpenalty = 10000
\widowpenalty = 10000
\displaywidowpenalty = 10000

%Gummi|065|=)
\title{Verhaltensneurogenetik}
\author{}
\date{}

% set title of table of contents
\renewcommand*\contentsname{Inhalt}

% https://www.sharelatex.com/learn
% http://www.math.ubc.ca/~cautis/tools/latexmath.html
% http://www.golatex.de/wiki/Kategorie:Befehlsreferenz
% https://en.wikibooks.org/wiki/LaTeX/Mathematics

\begin{document}

\begin{titlepage}

\maketitle
\thispagestyle{empty}
\end{titlepage}
\newpage

\begin{titlepage}
\tableofcontents
\thispagestyle{empty}
\end{titlepage}
\newpage

\section{Vorlesung 20.04.2016}

\subsection{Mechanorezeptoren von C. elegans}
\begin{itemize}
	\item 302 Neurone + 56 gliaähnlihce Zellen + 601 übrige Zellen = 959 Zellen gesamt + Keimbahn
	\item 6 Mechanorezeptoren (3 AVM, 3 PVM - Posterior ventral microtubule cell). PVM nicht funktionell. AVM hat Synapsen zu Motorneuronen. Bei Verschiebung an andere Position wird PVM funktionelle bzw. AVM nicht funktionell -$>$ keine rein zellautonome Entwicklung
	\item Fluchtreaktion nach mechanischem Stimulus (nach Mutagenese nicht mehr) -$>$ aber Kontrollversuch „Bewegung nach nicht-mechanischem Stimulus“ notwendig
	\item 350 Mutanten. Mutation in 18 Genen. Ca. 20 Allele / Gen. Keine Genkomplexe. Mutation betrifft immer alle 6 Zellen.
\end{itemize}

\textbf{Klassifikation der Mutationen:}
\begin{itemize}
	\item Zellstammbaumdefekte
	\item Determinationsdefekte
	\item Störung des sensorischen Transduktionsprozesses
	\item Spezifischer Zelltod
	\item Konnektivitätsdefekte
\end{itemize}

\subsection{Entwicklung von Nervensystemen (Neurogenese)}
Neuronale Stammzelle (Neuroblast) -$>$ Ganglienmutterzelle -$>$ Differenzierte Zelle (Neuron)\\
Kein Exponentielles Wachstum. Inäquale Teilung der Ganglienmutterzellen, so dass Neuronen unterschiedliche Komponenten bekommen\\
\\
Das \underline{Neuroektoderm} ist eine in der Embryonalentwicklung aus dem äußeren Keimblatt (Ektoderm) hervorgehende Bildung. Aus ihm entwickelt sich über das Neuroepithel des Neuralrohrs das zentrale, und über die Neuralleisten auch das periphere Nervensystem.\\

\underline{Oogenese}, auch Ovogenese, von lat. ovum, das Ei, ist die Entwicklung einer befruchtungsfähigen Eizelle (Ovum) aus einer Zelle der Keimbahn bei mehrzelligen Tieren.

Entdeckt wurden diese Gene anhand der Antennapedia-Mutation eines \underline{homeotische Gens} bei dem Modellorganismus Drosophila melanogaster, bei der am Fliegenkopf anstelle von Antennen Beine wachsen. Es handelt sich dabei um Gene, die regulatorische Proteine codieren.

\subsection{Neurogenese in der Entwicklung von Drosophila}
\underline{Positionsinformation:}
\begin{itemize}
	\item anterior-posterior/ ventral-dorsal -Gradienten, (z.B. bicoid)
	\item Segmentierung; hierachisches System von Transkriptionsfaktoren
	\item Homeotische Gene
\end{itemize}

\underline{Neuroectoderm}
\begin{itemize}
	\item Procephale Region: Hemisphere des larvalen Gehirns
	\item Ventrale neurogene Region: Segmentierter Teil des ZNS. Nicht streng zellautonom (Ablation einzelner Zellen führt nicht zum Verlust von Neuroblasten). Primäres Entwicklungsziel: Neuroblast
\end{itemize}

\underline{Segregation ins Innere des Embryos}
\begin{itemize}
	\item Segregation über 3h in Wellen
	\item Beginn: morphologische Veränderung der Zelle erkennbar (groß, rund)
	\item Neuroblasten teile sich dann ca. 8-9x
	\item Im 1. Larvenstadium Größenzunahme und Bildung der Neuronen.
\end{itemize}

Determination und Differenzierung durch zwei Gruppen von Genen gesteuert: proneurale und neurogene Gene\\

\underline{Proneurale Gene} (verleihen den Zellen das primäre neuronale Entwicklungspotential):
\begin{itemize}
	\item Determination der Zellen der neurogenen Regionen; Verlust der Gene bewirkt den Verlust je nach Gen unterschiedlicher Teile des Nervensystems (Hypoplasie).
	\item Beispiele:
		\begin{itemize}
			\item Achaete-scute-Komplex: 25\% Verlust an Neuroblasten
			\item Daughterless: nicht ausschliesslich proneurale 				\item Funktion sondern auch maternale und zygotische Expression
		\end{itemize}
\end{itemize}

\underline{Neurogene Gene} (1800 Zellen, aber nur 450 werden zu Neuroblasten; Rest epidermale Sturkturen -$>$ Funktion besteht darin sicherzustellen, dass keine übermäßige Neuronenbildung):
\begin{itemize}
	\item Mutationen dieser Gene führen zur Neuralisierung des Embryos (Hypertrophie des NS auf Kosten epidermaler Strukturen)
	\item Epistatische (wenn ein Gen die Unterdrückung der phänotypischen Ausprägung eines anderen Gens bewirken kann) Wechselwirkungen demonstrieren die funktionelle Hierarchie der neurogenen Gene
	\item Laterale Inhibition durch Notch-vermittelte Signaltransduktion (erst bilden beide Zellen Notch-Rezeptor und DSL-Liganden; dann wird bei einer DSL-Ligandenexpression (Delta) und bei der anderen Produktion Notch-Rezeptor herabgesetzt). Oder auch Signal Delta und Notch-Rezeptor
	\item Notch ist ein Universalempfänger für Zellkommunikation (nicht spezifisch für neurogene Region). Notch-Gen: 40kb, komplex reguliert, 10.5kb mRNA, posttranslationale Modifikation
	\item Notch ist Transmembranprotein mit einer extrazellulären und einer intrazellulären Domäne
	\item Nachweis Notch-Delta-Wechselwirkung: Zellaggregation in Zellkulturen
	\item Notchregulierung durch Membranprotein Numb
	\item Delta auf Seite der Neuroblasten und Notch auf Seite der Epidermoblasten; Zelle ohne Notch -> Epidermoblasten; ohne extrazellurlären Notchteil -$>$ Neuroblasten
	\item Bindung der DLS Liganden an Notch Auslöser für proteolytische Signalkaskade
\end{itemize}

MUTANTEN: Notch, Delta, Achaete-scute-Komplex (proneural), big brain, hairless\\

Neuroblasten teilen sich inäqual in einen Neuroblasten und eine Ganglienmutterzelle. Ganglienmutterzelle kann sich noch einmal teilen und produziert Neurone oder Gliazellen. Wenn numb-Protein, dann Neuron. Neuroblaten geben jedesmal andere Transkriptionsfaktoren an die Ganglienmutterzellen.\\

\textbf{Struktur – Funktionsbeziehung am Beispiel des Central Complex von Drosophila
}\\

Protocerebral bridge / Fan shaped body / Ellipsoid body / Noduli

\underline{Buridians Paradigma:}\\
Auch in diesem Experiment läuft die Fliege zunächst auf einen der beiden Balken zu. Auf dem Weg dorthin hält sie jedoch der Wassergraben auf, den die Fliegen vermeiden. Da dieser Balken unerreichbar ist, entscheidet sich die Fliege um und läuft stattdessen auf den zweiten Balken zu, doch auch dort kommt die Fliege dem Balken nicht näher als beim ersten Versuch am anderen Ende der Plattform. Also dreht sie wieder um und versucht erneut den ersten Balken zu erreichen. Dort angekommen spielt sich wiederum das gleiche Spiel ab und so fort. In Anlehnung an Buridans Esel nennen Neurobiologen dieses Experiment „Buridan Paradigma“. Zuerst 1982 beschrieben, war dieses Experiment vor allem dazu gedacht, etwas über das optische System der Fliegen zu lernen und herauszufinden wie sie dieses zur Kurskontrolle verwenden.
\\\\
\underline{Alternativ: Modifizierter Detour-Aufbau}\\
Anderer Versuch: Flugsimulator\\
rut: rutabaga, codiert Ca/Calmodulin-abhängige Adenylatcyclase, „klass.“ Lernmutante\\
rut2080: Protein notwendig für wildtypischesLernverhalten\\
hier: rutabaga-Protein in spez. Zellen hinreichend für Lernverhalten\\
Mutante hat Störung des Kurzzeitgedächtnis, aber auf erniedrigtem Niveau gute Persistenz des Langzeitgedächtnis -> hervorgerufen durch Störung des cAMP Stoffwechsel (Cyclisches Adenosinmonophosphat)

\end{document}