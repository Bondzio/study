\section{V9}
\subsection{Interpretation der Fixationsindices F$_{st}$ und F$_{is}$}
Fixationsindizes (F-Statistiken) sind Maße der genetischen Variabilität (Varianzkomponenten).\\

\textbf{Verursacht durch evolutionäre Prozesse:}
\begin{itemize}
	\item Inzucht
	\item Migration
	\item Mutation
	\item Selektion
	\item Wahlund Effekt\footnote{Die Vereinigung von zwei Populationen mit unterschiedlicher Allelfrequenz verringert die Heterozygosität im Vergleich zur HWE Erwartung.\url{https://de.wikipedia.org/wiki/Wahlund-Effekt}}
\end{itemize}

Fortpflanzung zwischen Verwandten Individuen führt zu einer Reduktion der Heterozygosität (im Vergleich zu HWE) und wird gemessen durch den Inzuchtskoeffizienten F\\
Zusammenhang zwischen F und Genotyp-Häufigkeiten für zwei Allele:\\
$p(A)=p(AA) + \frac{p(AB)}{2}$, $p(B)=1-p(A)$\\
$p(AA)=p(A)^2(1-F) + p(A)F$\\
$p(AB)=2p(A)p(B)(1-f)$\\
$p(BB)=p(B)^2(1-F) + p(B)F$\\\\
$\Rightarrow$ $\displaystyle F=1-\frac{O(AB)}{E(AB)}=1-\frac{p(AB)}{2p(A)p(B)}$\\
F hängt offenbar von der Allelfrequenz einer Bezugs-Population ab
\\\\
$F_{IS}$ = Inzuchtkoeffizient eines Individuums I relativ zu Subpopulation S\\
$F_{ST}$ = Maß für den genetischen Abstand einer Subpopulation S zur Gesamtpopulation T\\
$F_{IT}$ = Inzuchtkoeffizient eines Individuums I relativ zur Gesamtpopulation T

\newpage
Man kann die Fixationsindizes auch als Varianzkomponenten
modellieren:
\begin{itemize}
	\item $F_{IS}$:
	\begin{itemize}
		\item entspricht der Varianzkomponente des Innersubjektfaktors innerhalb einer Subpopulation
		\item $\displaystyle F_{IS}=1-\frac{\text{observed heterocygosity}}{heterocygosity under HWE}$
	\end{itemize}
	\item $F_{ST}$
	\begin{itemize}
		\item entspricht der Varianzkomponente der Populationssubstruktur innerhalb der Gesamtpopulation
		\item $\displaystyle F_{ST}=1-\frac{\sigma^2_{\text{within populations}}}{\sigma^2_{\text{between populations}}}$
	\end{itemize}
	\item $F_{IT}=F_{IS} + F_{ST} - F_{IS}F_{ST}$
\end{itemize}

\subsection{Bootstrap, Jackknife als Schätzverfahren für Standardfehler}
\textbf{Bootstrap}\\
\begin{itemize}
	\item Sei X eine Zufallsstichprobe, S eine interessierende Statistik auf X
	\item Betrachte die empirische Verteilung auf X
	\item Ziehe n=\#X Zufallsstichproben aus X (mit Zurücklegen)
	\item Berechne S für diese Zufallsstichproben
	\item Wiederhole Ziehung der Zufallsstichproben und Berechnung von S
	\item Bestimme die Standardabweichung für diese Ergebnisse
	\item Diese Standardabweichung ist ein Schätzer des Standardfehlers
von S auf X
\end{itemize}

\textbf{Jackknife}
\begin{itemize}
	\item Sei X eine Zufallsstichprobe, S eine interessierende Statistik auf X
	\item Betrachte die n=\#X Teilstichproben $X_i$ die man erhält, wenn man einzelne Elemente wegläßt
	\item Berechne S für diese Zufallsstichproben $\rightarrow$ $S_i$
	\item Berechne $\displaystyle \hat{\mu}=\frac{\sum_{i}^{n}S_i}{n}$ und $\hat{\sigma} = \sqrt{\frac{n-1}{n} \sum_{i}^{n} (S_i-\hat{\mu})^2}$
	\item Dies ist ein Schätzer des Standardfehlers von S auf X
\end{itemize}

\subsection{Hauptkomponentenanalyse in der Genetik (Interpretation)}
\begin{itemize}
	\item “Structure” erkennt nur grobe genetische Unterschiede zwischen Populationen
	\item Hauptkomponentenanalyse vieler SNPs ist (viel) sensitiver
	\item Hauptkomponenten erlauben es, Genetische Assoziationen auf Populationsstruktur zu korrigieren
\end{itemize}

Vorgang:
\begin{enumerate}
	\item Markermatrix normalisieren: $M(i,j)=\frac{C(i,j) - \mu(j)}{\sqrt{p(j)(1-p(,)}}$ \textcolor{red}{???}
	\item Wishart-Matrix aufstellen: $X=\frac{1}{n}MM'$
	\item Eigenwerte berechnen und ordnen: $\lambda_1 > \lambda_2 > ... > \lambda_m$
\end{enumerate}

Die hohe Empfindlichkeit der Methode erfordert eine sorgfältige Analyse:
\begin{itemize}
	\item Angleichung der Fallzahlen der Cluster
	\item Sorgfältiger Umgang mit Ausreißern
	\item Strenge SNP-Qualitätskriterien
	\item SNPs vor Analyse entkorrelieren (pruning)
	\item Bekannte, kritische genomische Bereiche (konservierte Bereiche oder stark mit Abstammung assoziierte Bereiche) eliminieren
	\item Mit PCs assoziierte SNPs + Regionen eliminieren
	\item Funktioniert nicht ohne weiteres mit verwandten Individuen (Lösung: „Drop-one-in“ Prozedur, Alternative: „swap-one-in“)
\end{itemize}

\newpage
\subsection{ROH: Definition und Interpretation}
\textbf{Idee:} Längere homozygote Bereiche im Genom weisen auf (ent-
fernte) Verwandtschaft zwischen den Eltern hin (inbreeding).
\\\\
Tritt das Phänomen in einer Population gehäuft auf, so spricht dies
für einen kleinen Genpool (kleine Gründerpopulation = founder),
bzw. geringen Austausch mit der Umgebung = Isolation\\

Praktische Probleme mit der Definition (SNP-Arrays):
\begin{itemize}
	\item Genotypisierungsprobleme: einzelne, möglicherweise falsche heterozygote oder fehlende Genotypen in einem Bereich
	\item Wo beginnt, wo endet ein homozygoter Bereich?
	\item Die Chance auf Homozygotie hängt von der Anzahl der SNPs in einem Bereich bzw. der SNP-Dichte und den Allelfrequenzen ab
	\item Ab welcher Entfernung „teilt“ man homozygote Bereiche?
	\item Lösung: Hidden Markov Modelle
\end{itemize}

\subsection{Aufgaben zur Übung 9}
\subsubsection{Aufgabe 1}
\textbf{a.)} Was sind Batch-Effekte?\\
eine technische Quelle für Variation in den Daten durch die Verarbeitung\footnote{\url{http://www.molmine.com/magma/global_analysis/batch_effect.html}}
\\\\
\textbf{b.)} Durch was können sie entstehen, wie kann man sie vermeiden?\\
mögliche Quellen:
\begin{itemize}
	\item \textbf{Spotting:} Die Menge der Probe in den Nadeln des Roboters, der damit das Array behandelt, kann leicht variieren.
	\item \textbf{PCR Amplikation:} Proben, die durch die Polymerase-Kettenreaktion(PCR) erzeugt werden, enthalten oft nicht die gleichen Vielfachen einer Sequenz, da die Amplikation der unterschiedlichen Nukleotidstränge mit unterschiedlicher Geschwindikeit verlaufen kann.
	\item \textbf{Probenaufbereitung:} bei der Vorbereitung der Proben ist eine Vielzahl komplexer biochemischer Reaktionen, wie zum Beispiel die reverse Transkription, durchzuführen. Diese können von Labor zu Labor und innerhalb eines Experiments Unterschiede aufweisen.
	\item \textbf{RNA-Abbau:} Unterschiedliche RNA-Stränge haben aufgrund ihrer Sekundärstruktur eine unterschiedliche Halbwertszeit. Um sie zu stabilisieren, werden eine Vielzahl von Gegenmaßnahmen angewendet, die auch Nebeneffekte nach sich ziehen können.
	\item \textbf{Array-Beschichtung:} Sowohl die Effizienz der Probenfixierung auf dem Array, als auch die Intensität des Hintergrundrauschens hängt stark von der Array-Beschichtung mit der Probe ab.
\end{itemize}
Diese Probleme sollten beim Design eines Mircoarray-Experiments beachtet werden. Kann man trotz allem einen Fehler nicht verhindern, so sollten die experimentellen Bedingungen so gewählt werden, dass die biologische Fragestellung nicht beeinflusst wird. Falls zum Beispiel ein Vergleich zwischen zwei Tumorprob en durchgeführt werden soll, so ist es ratsam, beide Prob en nicht in verschiedenen Labors aufbereiten zu lassen.\footnote{\url{http://www-stud.rbi.informatik.uni-frankfurt.de/~linhi/SeminarSS04/Ausarbeitungen/03ausarbeitung_evgenji_yusuf.pdf}}
\\\\
\textbf{c.)} Erinnern Sie sich an Aufgabe 4 von Blatt 6. Statt verschiedener Populationen nehmen wir nun an, dass der SNP auf verschiedenen Platten gemessen wurde. Führen Sie einen Chi-Quadrat-Test durch, ob sich die Allelhäufigkeiten zwischen den Platten signifikant unterscheidet!\\\\
Ergebnisse siehe R-Skript