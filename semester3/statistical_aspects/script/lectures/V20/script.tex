\section{V20}
\subsection{eQTL}
\begin{itemize}
	\item \textbf{E}xpression \textbf{Q}uantitativ \textbf{T}rait \textbf{L}ocus
	\item Genetik des Transkriptoms“ = GWAS für Genexpressionen
	\item Welche genetischen Faktoren ($\sim 10^7$) korrelieren mit welchen RNA-Transkripten ($\sim 10^4$)?
	\item eQTL = Schnittmenge zwischen Phänotyp, genetischen Faktoren und Genexpression
\end{itemize}

\textbf{Anwendung}
\begin{itemize}
	\item Funktioneller Relevanz / Validierung genetischer Hits (z.B. GWAS – Hits)
	\item Identifizierung neuer genetischer Risikofaktoren
	\item Aufklärung grundlegender biologischer Zusammenhänge bzw. Krankheitsmechanismen
\end{itemize}

\subsection{Cis/trans Effekte}
\begin{itemize}
	\item cis–eQTL:
	\begin{itemize}
		\item SNP korreliert mit benachbarter Expressionssonde
		\item Stärkere Effekte
		\item Massenphänomen aktuell bei mehr als der Hälfte aller Gene beobachtet
		\item Weniger falschpositive unmittelbare Hypothesen bzgl. kausalem Gen $\rightarrow$ besonders relevant für Follow up
	\end{itemize}
	\item trans–eQTL:
	\begin{itemize}
		\item SNP korreliert mit distaler Expressionssonde
		\item Teilweise bedeutend geringere Effektstärke
		\item Erfordern größere Studien
		\item Liefern Einblicke in die genetische Regulation (nc-RNA, Epigenetik)
		\item Könnten GWAS-Hits in „Genwüsten“ erklären
	\end{itemize}
\end{itemize}

Trennung ist schwammig und uneinheitlich, da „benachbart“ schlecht definierbar ist. Wir verwenden meist 1MB.

\subsection{Bedeutung von eQTLs}

\subsubsection{Für Verständnis der Regulation der Genexpression}
\textcolor{red}{???}
\subsubsection{Zur Erklärung genetischer Assoziationen}
\textcolor{red}{???}
