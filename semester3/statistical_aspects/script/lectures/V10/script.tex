\section{V10}
\subsection{Heritabilität, Definition + Möglichkeiten zur Schätzung}
Erklärte Varianz eines Merkmals durch Genetik.\\
Schätzung der Vererbbarkeit:
\begin{enumerate}
	\item Zwillingsstudien: Vergleich der Merkmalskonkordanz zwischen ein- (MZ) und zweieiigen Zwillingen (DZ)\\
	\textbf{Falconers Gleichung: $H^2=2r(MZ)-2r(DZ)$}
	\item Familienstudien: Gemischte Modelle, Verwandtschaft = Kovarianz
\end{enumerate}

\textbf{Schätzen der Heritabilität aus genomweiten Daten}\
gemischtes Modell: $y_i=\mu+G_i+e_i$ mit \\
$y_i$: Phänoptyp Induviduum i\\
$\mu$: Mittelwert\\
$G_i \sim \sigma^{2}_{Genetik}N(0,\Phi)$ (Polygenetischer Effekt)\\
$\Phi$: Verwandschaftsmatrix\\
$e_i \sim \sigma^{2}_{Umwelt}N(0,1)$ (Umwelteffekt)\\

\textbf{Heritabilitätsschätzung:} $H^2=\frac{\sigma^{2}_{Genetik}}{\sigma^{2}_{Genetik} + \sigma^{2}_{Umwelt}}$

\subsection{Genetische Assoziation (Prinzip)}
SNP dient als Stellvertreter (proxy) eines Krankheitslokus. SNP muß nicht selbst kausal sein, sondern es genügt LD mit kausaler Variante.\\\\
Variante A: direkte Kausalität\\
Variante B: indirekte Kausalität über LD und proxy

\subsection{Stratifikationsbias bei genetischen Studien}
Gedankenexperiment:
\begin{itemize}
	\item Phänotyp: Fähigkeit mit Stäbchen zu essen
	\item Studienpopulation: Gemisch aus Europäern und Asiaten
	\item Alle SNPs mit unterschiedlicher Allelfrequenz zwischen Europäern
und Asiaten (viele!) sind mit dem Phänotyp assoziiert $\rightarrow$ Inflation der
Teststatistiken $\rightarrow$ \textbf{Stratifikationsbias}
\end{itemize}

Andere Möglichkeit der Entstehung: Verwandtschaft zwischen den Individuen
\\\\
\textbf{Möglichkeiten der Bekämpfung von Stratifikationsbias (später genauer)}
\begin{itemize}
	\item Stratifizierte Analyse: Vorteil: funktioniert am besten, Nachteil: erfordert Kenntnis der Strata
	\item Adjustierung auf Hauptkomponenten: Vorteil: einfach durchführbar, Nachteil: Erhöht Freiheitsgrade der Assoziationsmodelle, beseitigt Stratifikation nicht immer vollständig, erfordert genomweite Daten (später)
	\item Local ancestry: Vorteil: Kann mit „Mischlingen“ umgehen, Nachteile wie bei „Adjustierung auf Hauptkomponenten“, erfordert genomweite Daten (später)
	\item Gemischte Modelle (Assoziation mit Korrelationsstruktur): Vorteil: geeignet bei Verwandtschaft, Nachteil: hohe Rechenbelastung
	\item Genomic Control: Einfache, phänomenologische Korrektur bei geringer Inflation, erfordert genomweite Daten (später)
\end{itemize}

\subsection{Genetische Modelle und deren Schätzung}
Allgemeines genetisches Modell ohne Kovariablen:\\
$y_i=\mu + \beta_{AB}\chi^i_{AB} + \beta_{BB}\chi^i_{BB} + e_i$ mit\\\\
$y_i$: Phänoptyp Induviduum i\\
$\mu$: Mittelwert\\
$\chi^i_{XX}=$ 1 falls Indiviuum i Genotyp XX hat, 0 sonst\\
$\beta$: Regressionskoeffizienten\\
$e_i \sim \sigma^{2}_{Umwelt}N(0,1)$ (Residualvarianz)\\
\\
\begin{tabular}{|c|c|c|}
\hline
	Modell & Vergleich & Tests\\
\hline
	Additiv & AA vs. AB vs. BB & $\beta_{AB}\neq 0, \beta_{BB}=2\beta_{AB}$\\
\hline
	Dominant B & AA vs. AB, BB & $\beta_{AB}\neq 0, \beta_{BB}=\beta_{AB}$\\
\hline
	Rezessiv B & AA, AB vs. BB & $\beta_{AB}=0, \beta_{BB}\neq 0$\\
\hline
	Heterozygotenvorteil & AA, BB vs. AB & $\beta_{AB}\neq 0, \beta_{BB}=0$\\
\hline
\end{tabular}

\begin{itemize}
	\item Additives Modell am flexibelsten
	\item Dominat B = Rezessiv A
	\item Heterozygotenvorteil selten (Beispiel Sichelzellanämie / Malaria)
	\item Problem: Genetisches Modell i.d.R. unbekannt
\end{itemize}

\newpage
\textbf{Strategien}
\begin{enumerate}
	\item 
	\begin{itemize}
		\item Man rechnet (trotz Unsicherheit über das tatsächliche Modell) mit nur einem (meist dem additiven) Modell
		\item Rationale:
		\begin{itemize}
			\item Modelle sind korreliert (Ausnahme Heterozygotenmodell)
			\item Bsp: liegt tatsächlich eine Assoziation unter dem dominanten/rezessiven Modell vor, wird dies höchstwahrscheinlich auch im additiven Modell sichtbar
		\end{itemize}
		\item Diese Strategie wird überwiegend angewendet
		\item Problem: Powerverlust bei falschem Modell, additives Modell erkennt Heterozygotenvorteil nicht
	\end{itemize}
	\item 
	\begin{itemize}
		\item Man vergleicht die Paßförmigkeit der Modelle für jede Situation (z.B. AIC), sucht sich das beste heraus und nimmt dessen p-Wert
		\item Problem: Dieses Verfahren inflationiert den Typ 1 Fehler $\rightarrow$ erfordert Permutationstests, diese sind aufwendig
	\end{itemize}
	\item 
	\begin{itemize}
		\item Man testet mehrere Modelle und korrigiert nach Bonferroni
		\item Problem: Überkonservativ, da Korrelation zwischen Modellen nicht berücksichtigt
	\end{itemize}
	\item Max-Test: ...
\end{enumerate}

\subsection{Spezifik gonosomaler Markeranalysen}
\begin{itemize}
	\item Die nPAR-Region ist bei Frauen auf einem Chromosom häufig inaktiviert.
	\item Die Inaktivierung ist wahrscheinlich(?) zufällig.
	\item Man modelliert die Inaktivierung indem man die Codierung der Genotypen der Männer entsprechend anpaßt:
\end{itemize}
$
\text{Codierung (Frau)} =
  \begin{cases}
    0  & \quad \text{für AA}\\
    1  & \quad \text{für AB}\\
    2  & \quad \text{für BB}\\
  \end{cases}
$

$
\text{Codierung (Mann)} =
  \begin{cases}
    0  & \quad \text{für A}\\
    c  & \quad \text{für B}\\
  \end{cases}
$\\
mit c=1: Keine Inaktivierung; c=2: Vollständige Inaktivierung; 1$<$c$<$2: Unvollständige Inaktivierung

\begin{itemize}
	\item Der Grad der Inaktivierung kann mit geschätzt werden (kompliziert)
	\item Man sollte unbedingt den Haupteffekt des Geschlechts in die Regressionsmodelle des X-Chromosoms einbauen
	\item Speziell bei Annahme keiner Inaktivierung, sollte die IA von Marker und Geschlecht mit ins Modell gesteckt werden.
	\item Obwohl Geschlecht ein unabhängiger Risikofaktor vieler Erkrankungen ist, heißt es nicht, dass auf Chr. X besonders viele Assoziationen zu finden sind (Beispiel: KHK)
\end{itemize}

\subsection{Genomweite Assoziationsstudie}
\begin{itemize}
	\item Ziel: Identifikation genetischer Modifikatoren beobachtbarer Phänotypen
	\item Hinweise für Vererbbarkeit aus z.B. Zwillingsstudien
	\item „Komplexe Erkrankungen“ $\rightarrow$ Polygenetische Effekte: Häufige Varianten mit geringer Penetranz, seltene Varianten mit höherer (?) Penetranz
	\item Kandidatengenansätze häufig nicht replizierbar $\rightarrow$ hypothesenfreie Ansätze $\rightarrow$ Screening des Genomes mittels Marker (SNPs)
	\item Aktuell ca. 2240 publizierte GWAS, mehrere hundert verschiedenen Phänotypen
\end{itemize}

\subsubsection{Ansatz}

\subsubsection{Replikation vs. kombinierte Analyse und Power}
\underline{Replikation:}
\begin{itemize}
	\item Einzelanalyse der Top-Marker aus der ersten Stufe (GWAS)
	\item Verwendet nicht die Evidenz aus der ersten Stufe

\end{itemize}

\underline{Kombinierte Analyse:}
\begin{itemize}
	\item Kombiniert die Information aus erster und zweiter Stufe
	\item Achtung! Effektgrößen aus der erster Stufe sind inflationiert („winners curse“) $\Rightarrow$ Korrektur (kompliziert)
\end{itemize} 

Die Power der kombinierten Analyse ist immer höher als die der
Replikation, aber Kombinierte Analyse hat nur dann (deutlich) höhere Power wenn:
\begin{itemize}
	\item Mehrzahl der Samples in Stufe 1 und
	\item Viele Marker in der Replikation
\end{itemize}

\subsubsection{Mehrstufendesign}
\begin{enumerate}
	\item GWAS: Genotype full set of SNP's in relatively small population at liberal p value
	\item Replikationsstufe: Screen second, larger population at more stringent p value
	\item optional third stage for increased stringency
\end{enumerate}

\subsubsection{Power}

\subsection{Aufgaben zur Übung 10}