\section{V11}
\subsection{Phänotyp, Genotyp-Phänotyp-Beziehung}
\textbf{Phänotyp:} Erscheinungsbild/Merkmale eines Organismus (Morphologisch, Physiologisch, Psychisch)
\begin{itemize}
	\item ererbt (Genotyp)
	\item erworben (Umwelt)
	\item akut (auf einen äußeren Reiz)
\end{itemize}

\textbf{Phänotyp - Bestimmung:}
\begin{itemize}
	\item Messbarkeit
	\item Reliabilität
	\item Validität
	\item Vergleichbarkeit zwischen Studien
\end{itemize}

\textbf{Intermediärer Phänotyp:} Liegt in der Kausalbeziehung zwischen Genetik und Zielphänotyp; Beispiel: Cholesterin = intermediärer Phänotyp für Arteriosklerose
\\\\
\textbf{Genotyp-Phänotyp-Beziehung}\\
Definition:
\begin{itemize}
	\item Eine genetische Veränderung (Mutation) ist ursächlich für den Phänotyp (Kausalität)
	\item Grad der Abhängigkeit des Phänotyps vom Genotyp wird gemessen durch Heritabilität
\end{itemize}
Mögliche Ursachen für Abweichungen von einer strengen Genotyp-Phänotyp Beziehung:
\begin{itemize}
	\item Phänokopie: Merkmalsausprägung aus anderer Ursache
	\item Phänotypische Plastizität: Modulierbarkeit durch Umwelteinflüsse
	\item Unvollständige Penetranz: Nichtausprägung trotz vorhandener Mutation, Kompensation eines Mechanismus
	\item Dramatyp: Phänokopie als Reaktion auf akutes Geschehen
\end{itemize}

\subsection{Reliabilität, Validität}
\textbf{Reliabilität:}\footnote{\url{https://de.wikipedia.org/wiki/Reliabilit\%C3\%A4t}}
\begin{itemize}
	\item Anteil der Varianz von Messwerten, der durch tatsächliche Unterschiede des Merkmals begründet ist
	\item Hängt eng mit der Reproduzierbarkeit von Messungen zusammen
	\item Intra-Rater (observer) Reliabilität vs. Inter-Rater (observer) Reliabilität\footnote{\url{https://de.wikipedia.org/wiki/Interrater-Reliabilit\%C3\%A4t}}
	\item grafische Darstellung mittels Bland-Altman Diagramm\footnote{\url{https://de.wikipedia.org/wiki/Bland-Altman-Diagramm}}
\end{itemize}

\textbf{Konkordanz-Korrelations-Koeffizient (CCC)}\\
geeignetes Maß zur Bewertung der Übereinstimmung zweier \textbf{\underline{quantitativer}} Merkmale\\
$CCC(X,Y)=\frac{2cov(X,Y)}{var(X) + var(Y)+(E(X)-E(Y))^2}$
\\\\
\textbf{Cohen‘s Kappa}\\
geeignetes Maß zur Bewertung der Übereinstimmung zweier \textbf{\underline{binärer}} Merkmale\\
$\kappa=\frac{p_{00}+p_{11}-p_{0.}p_{.0}-p_{1.}p_{.1}}{1-p_{0.}p_{.0}-p_{1.}p_{.1}}$
\\\\

\textbf{Validität:}
\begin{itemize}
	\item Aussage zur Belastbarkeit einer Messmethode oder Operationalisierung. Wird tatsächlich das gemessen, was gemessen werden soll?
	\item Vergleich mit Goldstandard
\end{itemize}

\subsection{(Genetische) Studiendesigns}
\subsubsection{Querschnittstudien}

\begin{itemize}
	\item „Cross-Sectional Study“\footnote{\url{https://de.wikipedia.org/wiki/Querschnitt_(empirische_Forschung)}}
	\item Untersucht gesamte Population oder repräsentative Zufallsstichprobe
	\item Momentaufnahme zu gegebenem Zeitpunkt: Analysiert Prävalenzen (kann nicht zwischen Effekt auf Inzidenz oder Dauer unterscheiden)
	\item Ungünstig für seltene Phänotypen
	\item Besonderheiten der Stichprobenziehung in Analysen berücksichtigen
	\item „Selective Survival Bias“
	\item Einfache Durchführbarkeit, sehr häufiger Studientyp
\end{itemize}

\subsubsection{Kohortenstudien}
\begin{itemize}
	\item „Cohort study“\footnote{\url{https://de.wikipedia.org/wiki/Kohortenstudie}}, Längsschnittliche Studie (longitudinal study“)
	\item Beobachtung des Auftretens eines Zielmerkmals in einer Population über einen gewissen Zeitraum
	\item Population ist initial frei vom Zielmerkmal
	\item Exposition wird anfänglich gemessen
	\item Anreicherung seltener Expositionen möglich
	\item Analysiert Risikofaktoren für Inzidenzen
	\item Aufwändig (Zeit und Geld)
	\item Probleme mit drop-out beim follow-up
\end{itemize}

\subsubsection{Fall-Kontroll-Studien}\footnote{\url{https://de.wikipedia.org/wiki/Fall-Kontroll-Studie}}
\begin{itemize}
	\item Definierte Fälle (mit Merkmal) und Kontrollen (ohne Merkmal)
	\item Definition einheitlicher Ein- und Ausschlusskriterien sehr wichtig
	\item Analysiert Expositionseffekte
	\item Vorsicht mit Confounding und Stratifizierung!
	\item Selective survival bias, differential recall bias
	\item Keine direkte Bestimmung des Relativen Risikos
	\item Fall-Kontroll-Matching kann schwierig sein
\end{itemize}

\subsection{GxE Interaktion}
Gene–environment interaction\footnote{\url{https://en.wikipedia.org/wiki/Gene\%E2\%80\%93environment_interaction}}

\subsection{Coverage von Microarrays}
Maßzahl für die „Qualität“ des Inhalts eines Microarray-Produkts\\
Anteil der Referenz, die in hinreichend hohem LD ($r^2$) mit SNPs auf dem Microarray sind. Hängt ab von:
\begin{itemize}
	\item Referenz (meist HapMap, 1000Genomes, verschiedene Panels)
	\item Ethnie (z.B. für afrikanische Populationen Coverage i.d.R. viel schlechter)
	\item gewünschtem LD-Niveau
	\item cut-off für seltene Varianten
\end{itemize}

\subsection{Aufgaben zur Übung 11}