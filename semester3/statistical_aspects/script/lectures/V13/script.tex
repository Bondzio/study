\section{V13}
Weitere Sample-Filter beruhen auf:
\begin{itemize}
	\item Geschlechts-Analyse
	\item Verwandtschafts-Analyse
	\item Hauptkomponenten-Analyse (PCA nach den englischen „principal component analysis“)
\end{itemize}

\subsection{Interpretation X-Y Intensitätsplots}
\textbf{Geschlechts-Analyse}\\
\begin{itemize}
	\item notwendig da zwei Quellen für das Geschlecht eines Probanden zur Verfügung stehen:
	\begin{itemize}
		\item Probandendatenbank
		\item Geschlechtsbestimmung beim Calling (computed gender)
		\item Problem: computed gender wird (bei Affymetrix) nur über Heterozygosität des X-Chromosoms bestimmt
	\end{itemize}
	\item Regel: DB-Geschlecht $\neq$ calling-Geschlecht $\rightarrow$ Proband filtern (obwohl mitunter auch DB-Eintrag falsch sein kann)
	\item weiteres Problem: beim Calling gibt es drei Geschlechtseinstufungen: female, male und unknown (für unknown Entscheidung anhand eines X-Y-Intensity-Plots \& DB-Geschlecht)
\end{itemize}

\textbf{Unregelmäßigkeiten in X-Y-Intensity-Plots}\\
\begin{itemize}
	\item YYX-Männer
	\item Frauen mit höherer Y-Intensität
	\item Monosomie X
	\item ungewöhliche X-Heterozygosität (Poly-X-Frauen)
	\item Geschlechtswidersprüche (gemessen vs. Datenbank)
\end{itemize}

\subsection{Interpretation PCA}
\begin{itemize}
	\item vergleiche Grundlagen
	\item kann zur Plausibilisierung der Daten genutzt werden (Vergleich mit Referenz-Populationen)
	\item Identifikation von Ausreißern (ethnisch, schlechte Genotypisierung)
	\item Plausibilisierung von ethnischen Angaben (DB)
	\item Achtung - PCA interagiert mit Verwandtschaft!
	\begin{itemize}
		\item verwandte Individuen sind i.d.R. PCA – Outlier
		\item Lösung: „Drop one in procedure“ – sehr aufwendig
	\end{itemize}
\end{itemize}

\textbf{kann folgendes darstellen:}
\begin{itemize}
	\item Ethnische Ausreißer
	\item Batch-Effekte
	\item Plattform-Effekte
	\item Substrukturen in der Kohorte
\end{itemize}

\subsection{CNV Detektion mit SNP-Array und Interpretation von R-Ratio und B-Allelfrequency plots}
\textbf{Problem}
\begin{itemize}
	\item CNV-Bestimmung aus Microarrays arbeitet direkt mit den Allel-Intensitäten
	\item kurze CNVs nur über spezielle CNV-Sonden detektierbar (z.B. Affymetrix SNP 6.0)
	\item mittellange CNVs i.d.R. auch schlecht detektierbar, Ergebnisse stark von Plattform abhängig
	\item nur sehr lange CNVs sind gut detektierbar (z.B. Tumor-DNA)
	\item SNP-Arrays nur bedingt für Keimbahn-CNV-Analyse geeignet (besser Sequenzierung)
\end{itemize}

\underline{\textbf{R Ratio:}} Beobachtete Intensität / Referenzintensität, bei höherer Intensität Duplikation, bei geringerer Deletion\\\\
\underline{\textbf{B-Allelhäufigkeit:}} Anzahl der Banden ergeben Ordnung der CNV (z.B. Anzahl Duplikationen)