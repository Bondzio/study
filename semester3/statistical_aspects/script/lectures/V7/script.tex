\section{V7}
\subsection{Motivation und Ansatz für gemischte Modelle}
Das lineare Modell stößt bei machen Analysen an seine Grenzen, z.B wegen:
\begin{itemize}
	\item Fehlen der Linearität in den realen Daten $\rightarrow$ nichtlineare Regression
	\item Verletzung der Verteilungsannahmen an die zufälligen Fehler
	\item Stichprobe besteht aus Subgruppen, für die jeweils „eigene“ $\beta$‘s geschätzt werden müssten. D.h. die Grundannahme der Unabhängigkeit der Beobachtungen ist verletzt.
\end{itemize}

\textbf{Typische Beispielsituationen}
\begin{itemize}
	\item Longitudinale Daten (Im Rahmen von Längsschnittstudien werden in zeitlicher Abfolge wiederholte Beobachtungen an denselben Subjekten -unter möglicherweise verschiedenen Bedingungen- erhoben)
	\item Clusterdaten (Aus Primäreinheiten (Cluster) werden mehrere Individuen ausgewählt und die interessierenden Variablen erhoben: Kinder aus Familien, Schüler auf Schulen)
	\item Hierarchische Daten (z.B. Kontinent $\rightarrow$ Land $\rightarrow$ Stadt)
	\item In der Genetik: \textcolor{blue}{Verwandtschaft}
\end{itemize}

\textbf{Warum hier nicht lineare Regression?}
Hauptproblem ist hier die Korrelation in den Daten:
\begin{itemize}
	\item Fehlerterme in linearer Regression sind nicht mehr unabhängig
	\item beta-Schätzer in linearer Regression können die Struktur der Daten nicht angemessen berücksichtigen
\end{itemize}

\underline{Longitudinaldaten:} Im Rahmen von Längsschnittstudien werden in zeitlicher Abfolge wiederholte Beobachtungen an denselben Subjekten (unter möglicherweise verschiedenen Bedingungen) erhoben.
\\\\
\underline{Clusterdaten:} Aus Primäreinheiten (Cluster) werden mehrere Individuen ausgewählt und die interessierenden Variablen erhoben.
\\\\
Bei Longitudinal- und Clusterdaten lässt sich ein typische Datensatz (Stichprobe) wie folgt notieren: $(y_{i1}, ... , y_{in_i}, x_{i1}, ..., x_{in_i})$, i=1,...,m

\begin{itemize}
	\item m die Anzahl der Individuen bzw. Cluster
	\item $n_i$die Anzahl der Messzeitpunkte für Individuum i bzw. die Anzahl der Elemente im Cluster i
	\item Longitud.Daten: $y_{ij}$ ist der Wert der Zielgröße von Subjekt i zum Zeitpunkt $t_{ij}$
	\item Cluster: $y_{ij}$ ist der Wert der Zielgröße vom j-ten Element in Cluster i
\end{itemize}

\underline{Beachte:} Die Daten vom gleichen Subjekt/Cluster neigen oft dazu ähnlicher zu sein als die Daten verschiedener Subjekte/Cluster.
\\\\
\textbf{Ziele:}
\begin{itemize}
	\item Schätzung der Subjekt- /Clusterspezifischen Effekte
	\item Schätzung der Populationsspezifischen Effekte
	\item Schätzung der Korrelationsstruktur
\end{itemize}

\subsection{Feste und zufällige Effekte}
\textbf{Variabilitätsquellen}
\begin{itemize}
	\item Zwischen den Subjekten/Clustern, d.h. Abweichungen eines Subjekts/Clusters vom Populationsmittel.
	\item Innerhalb des Subjekts/Clusters, d.h. Abweichungen einer Messung vom Mittelwert des entsprechenden Subjekts/Clusters.
\end{itemize}

\textbf{\underline{Lösung: Zwei - Stufen - Verfahren oder lineares gemischtes Modell}}

\newpage
\textbf{Zwei-Stufen-Verfahren:}
\begin{enumerate}
	\item Stufe: Das subjektspezifische Modell $Y_i=Z_i\beta_i +  \epsilon_i$
	\begin{itemize}
		\item i aus \{1,...,m\} das Individium i
		\item $n_i$ Anzahl der Messwiederholungen bei Individuum i
		\item $Y_i=(Y_{i1}, ..., Y_{in_i})^T$ der $n_i$-dimensionale Vektor der Messungen von Individuum i
		\item $Z_i$ ist die $(n_i \times q)$-Matrix der (evtl. zeitlich abhängigen q Kovariablen $\beta_i$ (später random effects)
		\item $\beta_i$ ist der (unbekannte) q-dimensionale subjektspezifische Vektor der Effekte der Kovariablen
		\item $\epsilon_i=(\epsilon_{i1}, ..., \epsilon_{in_i})^T$ Fehlerterm mit $\epsilon_i \sim N(0, \sigma^2I_{n_i})$
	\end{itemize}
	\item Stufe: Das Modell $\beta_i=K_i\beta + b_i$ welches die Variabilität \underline{zwischen} den Individuen beschreibt
	\begin{itemize}
		\item i aus \{1,...,m\} das Individuum i
		\item $n_i$ Anzahl der Messwerte bei Individuum i
		\item $K_i$ eine $(q \times p)$-Matrix bekannter p Kovariaten auf
Populationsebene (später fixed effects)
		\item $\beta$ ist der p-dimensionale Vektor der unbekannten
Regressionskoeffizienten dieser Kovariaten
		\item $b_i \sim N(0,D)$ Fehlerterm mit Kovarianzmatrix D (also hier im Unterschied zur linearen Regression Fehler nicht mehr i.i.d.[independent and identically distributed])
	\end{itemize}
\end{enumerate}

Man will Modelle aus beiden Stufen gleichzeitig fitten $\rightarrow$ \textbf{Idee:} einsetzen der 2. Stufe in die 1. Stufe.
\\\\
\textbf{
Stufe 1: $Y_i=Z_i\beta_i + \epsilon_i$\\
Stufe 2: $\beta_i=K_i\beta + b_i$\\
Zusammen: $Y_i=Z_iK_i\beta + Z_ib_i + \epsilon_i$
}

\newpage
$\Rightarrow$ Umwandlung des hierachischen Modells in ein \\ \textcolor{blue}{\textbf{lineares gemischtes Modell}} (linear random effects model):
\\
$y_i=\underbrace{X_i\beta}_{\text{Feste Effekte}} + \underbrace{U_ib_i}_{\text{Zufällige Effekte}} + \underbrace{\epsilon_i}_{\text{Residuale Fehler}}$, i=1,...,m
\\\\
Hierbei sind:
\begin{itemize}
	\item Die zufälligen Effekte $b_i$ unabhängig \underline{identisch} normalverteilt (i.i.d.) mit Erwartung 0: $b_i \stackrel{i.i.d}{\sim} N(0,D)$
	\item D ist eine positiv semi-definite Matrix (Kovarianzmatrix) und beschreibt ggf. Abhängigkeiten zwischen den zufälligen Effekten
	\item $\epsilon_i$ unabhängig \underline{(aber nicht identisch!)} normalverteilt: $\epsilon_i \sim N(0,\Sigma_i)$
	\item $\Sigma_1,...,\Sigma_m$ individuelle Kovarianzmatrizen der Zeitreihen
	\item $b_1,...,b_m,\epsilon_1,...,\epsilon_m$ unabhängig
\end{itemize}

\textbf{Bemerkungen:}
\begin{itemize}
	\item Ein lineares gemischtes Modell (LGM) muss nicht aus einem 2-Stufen-Verfahren abgeleitet werden: Jedes 2 Stufen Verfahren führt zu einem LGM, aber nicht jedes LGM lässt sich aus einem 2-Stufen-Verfahren herleiten.
	\item Bei Longitudinaldaten können Kovariablen zeitlich variieren;
	\item Auch bei Clusterdaten können Variablen im gleichen Cluster variieren oder konstant sein
	\item Bis auf die Definitheit gibt es keine weiteren Annahmen an die Kovarianzmatrizen der zufälligen Effekte D und der Fehlersumme. Zusätzliche Annahmen hinsichtlich der Struktur der Matrizen ergeben Modelle verschiedener Komplexität und Flexibilität
\end{itemize}

\underline{Spezialfälle:}
\begin{itemize}
	\item \textbf{Conditional Independence:} \textcolor{red}{???}
	\item \textbf{Varianzkomponenten-Modell:} \textcolor{red}{???}
	\item \textbf{Random Intercept Modell:} Hier wird für jedes Individuum ein eigenes Intercept zugewiesen
	\item \textbf{Random Intercept – Random Slope Modell:} \textcolor{red}{???}
\end{itemize}

\newpage
\textbf{Abschließende Bemerkungen zum LGM: Warum also LGM‘s?}
\begin{itemize}
	\item Stichproben mit einer Substruktur verletzen typischerweise die für viele statistische Verfahren erforderliche Unabhängigkeit der Beobachtungen. Nichtbeachten dieser Substruktur führt zu fehlerhafter Ergebnissen (Tests, Konfidenzintervalle, ...)
	\item Abhängigkeiten zwischen Beobachtungen reduzieren den Informationsgehalt im Vergleich mit unabhängigen Beobachtungen.
	\item Beachtung individuenspezifischer Informationen kann Schätzgenauigkeit erhöhen
	\item Schätzer für zufällige Effekte ermöglichen individuelle Prognosen
	\item Ermöglichen Modelle mit fehlenden Werten
	\item Bei vielen Meßverfahren können mit Hilfe der LGMs Batch-Effekte berücksichtigt werden.
	\item In der Genetik sind diese Verfahren notwendig um Verwandtschaftsstrukturen zu berücksichtigen.
	\item Wichtig für die Kombination von Studien (Metaanalyse)
\end{itemize}

\newpage
\subsection{Idee der Hauptkomponentenanalyse}
\textbf{Ziele:}
\begin{itemize}
	\item Exploratives Untersuchen von (hochdimensionalen) Daten
	\item Visualisierung (hochdimensionaler) Daten
	\item Dimensionsreduktion
	\item Denoising (Rauschen der Daten mindern)
\end{itemize}
 
\underline{Zwischenziel:} Finde im m-dimensionalen Datenraum die Richtung größter Varianz
\\\\
\textbf{Pearsons Ansatz:} Welche Gerade passt am besten zu den Daten?
\begin{itemize}
	\item Die Gerade w, die das Rauschen minimiert und das Signal maximiert.
	\item Oder anders gesagt: Die Gerade, die die Varianz in den Daten maximiert
\end{itemize}

\textcolor{red}{Was hier noch???}

\subsection{Interpretation PCA-Plots und Eigenwerte}
\textcolor{red}{Und was hier so???}

\subsection{Aufgaben zur Übung 7}
