\section{DS6}
\underline{\textbf{Functional Eveness}}\\
Wie bei der Artendiversität reduziert eine Ungleichverteilung der Abundanzen auch die funktionelle Vielfalt\\
3 Schritte zur Berechnung:
\begin{enumerate}
	\item Minimum Spanning Tree ausrechnen: $EW_l=\displaystyle \frac{dist(i,j)}{A_i + A_j}$ mit \\
	dist(i,j) = Distanz zwischen Art i und j\\
	$A_i$ = Abundanz von Art i
	\item Normieren mit Gesamtlänge: $PEW_l=\displaystyle \frac{EW_l}{\displaystyle \sum_{l=1}^{S-1} EW_l}$ mit\\
	S = Artenzahl\\
	p = relative Abundanz / Deckung / Biomasse\\
	dist = Euklidische Distanz im Merkmalsraum\\
	\item Index (0 bis 1) = Vergleich mit dem Idealszenario: $PEW_l$ bei ungleichen $EW_l$ immer kleiner als $\frac{1}{(S-1)}$\\
	$FEve=\displaystyle \frac{\displaystyle \sum_{l=1}^{S-1} min(PEW_l,\frac{1}{S-1}) - \frac{1}{S-1}}{1-\frac{1}{S-1}}$
\end{enumerate}

Beispiel siehe Vorlesung
\\\\
\underline{\textbf{Functional Dispersion}}
\begin{itemize}
	\item \textbf{Muss:}
	\begin{itemize}
		\item Muss mit mehreren Merkmalen funktionieren
		\item Darf nicht kleiner werden, wenn eine Art dazu kommt (monotonicity)
		\item Darf nicht größer werden, wenn eine Art gedoppelt wird (twinning principle)
	\end{itemize}
	\item \textbf{Soll:}
	\begin{itemize}
		\item Unabhängig von Artenzahl
		\item Unabhängig von Functional Richness
		\item Sollte Abundanzverteilungen berücksichtigen
	\end{itemize}
	\item \textbf{Wäre schön:}
	\begin{itemize}
		\item Kann alle Datentypen nutzen
		\item Funktioniert mit fehlenden Daten
	\end{itemize}
\end{itemize}

\newpage
\underline{\textbf{Rao‘s Q}}
\begin{itemize}
	\item „Rao‘s Quadratic Entropy“, „Varianz der Distanzen zwischen Arten“
	\item $FD_Q=\displaystyle \sum_{i=1}^{S-1}\sum_{j=i+1}^{S} dist_{ij}(p_i,p_j)$
	\item mittlere funktionelle Ungleichheit zwischen zwei zufällig ausgewählten Individuen
	\item \textbf{Vorteile:}
	\begin{itemize}
		\item Berücksichtigt Abundanz
		\item arbeitet mit multiplen traits (multivariat)
	\end{itemize}
	\item \textbf{Nachteile:}
	\begin{itemize}
		\item Wenn alle Arten gleich abundant sind (unwahrscheinlich), dann ist FD Q beeinflusst durch die Merkmalsverteilungen und die Kovarianz zwischen den Merkmalen
		\item wenn viele traits zum gleichen Merkmalssyndrom gehören (z. B. tradeoffs), dann wird diese Funktion	überbewertet.
	\end{itemize}
\end{itemize}

\underline{\textbf{FDiv}}
\begin{enumerate}
	\item Auswahl aller Vortex-Arten V
	\item Berechnung des Zentroids (Schwerpunkts) nur (!) der Vortex-Arten
	\item Koordinaten des Zentroids: ($g_1$, $g_2$, $g_3$,..., $g_k$)
	\item Ausrechnen der euklidischen Distanz zwischen jeder Art und dem Zentroid
	\item Ausrechnen der mittleren Distanz
	\item Abweichungen vom Ring mit Radius dG nach innen und nach außen. Abundanz gewichtet
	\item Index: $FDiv=\displaystyle \frac{\Delta d + d\bar{G}}{\Delta |d| + d\bar{G}}$
	\item kann nur Werte zwischen 0 und 1 annehmen
\end{enumerate}

\underline{\textbf{FDis von Laliberté und Legendre}}
\begin{enumerate}
	\item Zentroid aller Arten ausrechnen 1
	\item Distanz zum Zentroid für jede Art ausrechnen
	\item FDis ist der Mittelwert dieser Distanzen
\end{enumerate}

Die Auswirkung der Abundanzen ist:
\begin{itemize}
	\item dass sich das Zentroid zu den häufigen Arten hin verschiebt
	\item dass die Abstände zu häufigen Arten ein höheres Gewicht bekommen
\end{itemize}

\newpage
\textbf{Wo ist die Biologie? Ein Wort der Vorsicht!}\\
Man sollte die Merkmale verstehen, die man benutzt und Fragen stellen:
\begin{itemize}
	\item Welche Rolle spielen sie in Bezug auf Nischendifferenzierung?
	\item Welchen Mechanismus repräsentieren Sie? (z. B. Licht-, Wasser-, Nährstoffakquise?)
	\item Habe ich irgendeine Hypothese, warum Diversität oder Eveness bezüglich der Merkmale für meinen Prozess (Koexistenz, Produktivität, etc.) relevant sein könnte?
\end{itemize}

\subsubsection{Wie komme ich von BD zu EF?}
„Y“ = Ökosystemfunktionen (*=„Ökosystem-Dienstleistungen“)
\begin{itemize}
	\item Biomasse produzieren *
	\item Feuergefahr hervorrufen
	\item Stabilität gewährleisten *
	\item Kohlenstoff festlegen *
	\item Erosion verhindern *
	\item Luftstickstoff fixieren *
	\item Wasser verdunsten
\end{itemize}

\textbf{Was ist das „X“ genau?}\\
\begin{itemize}
	\item Ein Artname
	\item Ein Liste von Arten die gemeinsam vorkommen
	\item manchmal mit relativen Häufigkeiten
\end{itemize}

\underline{\textbf{Arten versus Arteigenschaften?}}\\
\textbf{Was passiert?}\\
Ökosystemfunktion = f(Name$_i$)\\
Ökosystemfunktion = f($\Sigma$Namen$_i$)
\\\\
\textbf{Warum passiert es?}\\
Ökosystemfunktion = f(Wachstumsrate$_i$)\\
Ökosystemfunktion = f(D Wurzeltiefe$_{ij}$)\\

\newpage
\underline{\textbf{Biodiversität (X) \& Ökosystemfunktionen (Y)}}
\begin{itemize}
	\item Auch kurz „BD-EF“ oder „BEF“ genannt
	\item Was hat das mit Merkmalen zu tun?
	\begin{itemize}
		\item Funktionelle Merkmale spielen eine wichtige Rolle bei der Übersetzung von Information (BD) in ökologische Prozesse (EF)
		\item Oder anders: Die Merkmale bestimmen im Wesentlichen, wie sich die Anwesenheit einer Art auf die Prozesse im Ökosystem auswirkt
		\item Zugespitzt:
		\begin{itemize}
			\item Wüsste man alle relevanten Merkmale bräuchte man die Artzugehörigkeit nicht wissen.
			\item Kennt man sie nicht alle, wird es immer residuale „Arteffekte“ geben.
		\end{itemize}
	\end{itemize}
\end{itemize}

\underline{\textbf{Facetten von FD und deren Bedeutung}}\\\\
Vorhersagekraft für Ökosystemfunktionen gibt Hinweis auf Mechanismus\\
\begin{tabularx}{\textwidth}{p{0.33\textwidth}p{0.33\textwidth}p{0.33\textwidth}}
	$\rightarrow$ & Merkmalswert & $\rightarrow$\\
	\hline
	Mittelwert & Verteilung & Spanne\\
	\hline
	Functional \textbf{Identity} & Functional \textbf{Dispersion} & Functional \textbf{Richness}\\
	\hline
	Mittlere Merkmalsausprägung bestimmt EF & Varianz bestimmt EF & Extreme sind wichtig\\
	Mass ratio hypothesis, Selection effect & Komplementarität, Insurance-Hypothese & Functionelles Potential ist wichtig\\
	Art der wichtigen Merkmal geben Hinweis auf Mechanismus& & \\
\end{tabularx}

\begin{itemize}
	\item Abundanzverteilungen sind wichtig
	\item Funktioniert mit vielen Merkmalen (multivariat)
\end{itemize}

\subsubsection{Funktionelle Merkmale bei Pflanzen}
\begin{itemize}
	\item Physiologie
	\item Morphologie
	\item Demographie
	\item Ökosystem
\end{itemize}

\newpage
\textbf{Auf welche Merkmale trifft man?}
\begin{itemize}
	\item Angaben aus der \textbf{taxonomischen Literatur}
	\begin{itemize}
		\item z. B. Blattlänge, maximale Höhe, Samengröße, ...
		\item ursprünglich nur zu Unterscheidungszwecken ausgewählt
	\end{itemize}
	\item Einfache ökologische \textbf{Gruppierungsmerkmale}
	\begin{itemize}
		\item z. B. Winderverbreitung: ja/nein, Schattentoleranzklasse: 1-5, ...
		\item Oft sind Gruppen nur eine Vereinfachung, weil die Erfassung eines quantitativen Masses viel zu aufwendig wäre
	\end{itemize}
	\item \textbf{Gezielte vergleichende Studien} der funktionellen Ökologie
	\begin{itemize}
		\item z. B. SLA, Gefäßdurchmesser , N-Gehalte, Blattlebensdauer,
		\item Funktionelle Bedeutung a priori relativ klar (nicht immer), oft vergleichender Natur
	\end{itemize}
	\item \textbf{Diverse Prozessstudien} aus der Botanik, Pflanzenphysiologie, Agrarwissenschaften, usw.
	\begin{itemize}
		\item Sehr pezifische Merkmale: Quantum-Use-Efficiency, Ammoniumaufnahmekinetik, ...
		\item I.d.R nur wenige Arten, oft Nutzpflanzen
	\end{itemize}
\end{itemize}

\textbf{Dateneigenschaften}
\begin{itemize}
	\item \textbf{Beschreibung}
	\begin{itemize}
		\item Beispiel: „Art bildet auf nassen Standorten häufig ein Flachwurzelsystem aus“
	\end{itemize}
	\item \textbf{Kategoriale Variablen (Kategorien)}
	\begin{itemize}
		\item Feuertoleranz (ja/nein oder 1/0, Blütenfarbe (grün, gelb, rot, blau,...)
	\end{itemize}
	\item \textbf{Ordinale Variablen}
	\begin{itemize}
		\item Beispiele: Überflutungstoleranz (1-5), Wachstumsrate (gering, mittel, hoch oder 1,2,3)
		\item Eigenschaft: 2 ist mehr als 1, aber nicht doppelt so viel; nie negativ
	\end{itemize}
	\item \textbf{Ganzzahlige Variablen}
	\begin{itemize}
		\item Anzahlen: Chromosomen, Griffel, Samen pro Kapsel
		\item Eigenschaften: nie negativ, keine Dezimalstelle
	\end{itemize}
	\item \textbf{Kontinuierliche Variablen}
	\begin{itemize}
		\item SLA (97.5 cm$^2$ g$_dw^{-1}$ )
		\item Eigenschaft: 2 ist doppelt so viel wie 1, kann positiv oder negativ sein
	\end{itemize}
\end{itemize}