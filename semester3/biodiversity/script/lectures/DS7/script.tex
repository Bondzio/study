\section{DS7}
\textbf{Spezifische Probleme}\\
Beschreibung: Übersetzung in Variablen?
\begin{itemize}
	\item \textbf{Kategoriale Variablen:}
	\begin{itemize}
		\item n Kategorien müssen in n-1 Variablen (0-1-codiert) umgesetzt werden. Das verbraucht Freiheitsgrade und ist umständlich.
		\item Zwei Wissenschaftler, zwei Kategoriesysteme, Vereinheitlichung ist schwierig.
	\end{itemize}
	\item \textbf{Ordinale Variablen:} Häufig sehr subjektiv („Försterlatein“)
	\item \textbf{Kontinuierliche Variablen:}
	\begin{itemize}
		\item Einheiten und Definitionen (Beispiel SLA: ein- oder zweiseitig)
		\item Oft pseudo-metrisch, z. B. Maximal Alter von Bäumen (400 Jahre)
	\end{itemize}
\end{itemize}

\textbf{Sonstige Probleme mit Merkmalsdaten}
\begin{itemize}
	\item \textbf{Taxonomie:} Fehlbestimmungen, Synonyme, Tippfehler, Subspezies
	\item \textbf{Datenbeschreibungen} (Meta-Daten) in den Quellen:
	\begin{itemize}
		\item Fehlerangaben ja oder nein?
		\item Welche Methode wurde verwendet?
		\item Wie wurde aggregiert? (Mittelwert von Blättern einer Pflanze, von Blättern mehrere Pflanzen, usw.)
		\item Skalen: Wann, über welchen Zeitraum, auf welcher Fläche etc.
	\end{itemize}
	\item \textbf{Es fehlen Daten} („missing data“)
\end{itemize}

\textbf{Was ist überhaupt ein Merkmal?}
\begin{itemize}
	\item Alles was man an einem Organismus messen kann? Öffnungsweite der Stomata oder bei Tieren Blutzuckerspiegel? – Merkmal?
	\item Alles was man messen kann und wenig variabel ist?
	\item Nur die Größen, die invariabel sind?
\end{itemize}

\newpage
\textbf{Grade von Variabilität}
\begin{itemize}
	\item \textbf{Fester Wert}
	\begin{itemize}
		\item z.B. Ausschlagfähigkeit,  Anz. Chromosomen
		\item $Y_i$: Wert, Klasse
	\end{itemize}	
	\item \textbf{Intrinsisch variabel}
	\begin{itemize}
		\item Samengewicht, Ligningehalt Borke \%
		\item $\overline{Y_i}$: Mittelwert ($\pm$Fehler)
	\end{itemize}
	\item \textbf{Umweltgesteuert}
	\begin{itemize}
		\item Respirationsrate, $V_{cmax}$, Zersetzungsrate
		\item $\theta_i$*: Parameter ($\pm$Fehler)
	\end{itemize}
\end{itemize}

\textcolor{red}{\textbf{Reduktionistische Ansatz der merkmalsbasierten Forschung?}}\\
Plastizität ist lebensnotwendig\\
\textbf{Fazit}
\begin{itemize}
	\item Phänotypische Plastizität ist kein Problem wenn Ausmaß und Sensitivität der Reaktion ungefähr gleich ist zwischen den Arten
	\item Phenotypische Plastizität ist ein Problen wenn sich die Reihenfolgen ändern
\end{itemize}

\textbf{Neue Klasse:}
\begin{itemize}
	\item Merkmal = f(Umwelt)
	\item Plastizität = f(Merkmale)
\end{itemize}

\textbf{Merkmale: Breite der Anwendung}
\begin{itemize}
	\item Zur Artenbestimmung
	\item Merkmalsevolution, komparative, merkmalsbasierte phylogenetische Analysen
	\item Nischenanalyse (Rosenzweig‘s Nische; 1987)
	\item Ableitung von Pflanzenstrategien; Einteilung von funktionellen Gruppen
	\item Klimaproxy in paläoökologischen Studien
	\item Prädiktor für Pflanzenperformanz; Züchtungsmarker (traits, states \& rates)
	\item BD-EF Analysen (Diversitäts- und Identitätsfragen)
	\item Parameter in Modellen
\end{itemize}

\newpage
\subsubsection{„Response“ und „Effekt“}
\textbf{Wichtige Schlüsselfragen}
\begin{itemize}
	\item Welche Merkmale sind beides (response und effect)?
	\item Inwiefern sind response und effect traits miteinander korreliert?
	\item Kann ich über ein Verbindung von response und effect traits sowohl die Entstehung von Biodiversität als auch die Effekte von Biodiversität erklären?
\end{itemize}

\textbf{Zuordnung}
\begin{itemize}
	\item Welches Merkmal ist für welche Funktion bedeutsam?
	\item Wird eine Funktion durch mehrere Merkmale gesteuert?
	\item Kann ein Merkmal mehrere Funktionen steuern?
	\item Kennen wir alle relevanten Merkmale?
\end{itemize}

\textbf{Ökologische Filter}
\begin{itemize}
	\item \textbf{Verbreitungsfilter:} Samenquelle, Verbreitungsvektoren
	\item \textbf{Abiotischer Filter:} Bodeneigenschaften, Klima, Störungen
	\item \textbf{Biotischer Filter:} Konkurrenz, Herbivorie, Pathogene
\end{itemize}

\textcolor{red}{\textbf{Zwei Beispiele}
\begin{itemize}
	\item Auf welcher Integrationsebene (innerhalb der Hierarchie) wirkt ein Merkmal?
	\item Welche Einfluss hat der Merkmalseinfluss auf Teilprozesse für den Gesamtprozesse (Komplexität: Viele Merkmale x viele Prozesse)
\end{itemize}}

\textbf{Fazit}
\begin{itemize}
	\item Merkmal beeinflussen im Beispiel die Ökosystemfunktion Monokulturbiomasse über unterschiedliche Pfade
	\item Staturmerkmale wirken über die individuelle Performanz ($\rightarrow$ ecosystem volume capture)
	\item Blatt- und Wurzelmerkmale wirken direkt ($\rightarrow$ biotische Interaktionen?)
	\item Die Merkmale wirken eher nicht über den Pfad der Populationsdichte, wenn dann die Wurzelmerkmale ($\rightarrow$ Packungsdichte evt. weniger wichtig)
\end{itemize}