\section{DS11}
\subsubsection{Ecosystem functions = f(Biodiversity)}
\textbf{Erste Hypothesen}
\begin{itemize}
	\item Seit den 1990er Jahren
	\item Extinktionsszenarien
	\item Drei Gruppen von Erklärungsmodellen:
	\begin{itemize}
		\item Arten sind redundant in ihrer Funktion
		\item Arten sind einzigartig in ihrer Funktion (Schlüsselarten, Ökosystem-Ingenieure)
		\item Arteneffekte sind kontext- abhängig und daher “idiosynkratisch”
	\end{itemize}
\end{itemize}

\textbf{Forschungsansätze:} Vergleichende Studien
\\\\
\textbf{Problem: Diversität = f(Umwelt)}
\\\\
\textbf{Also, Entkopplung von BD und Umwelt}
...um kausalen Zusammenhang zwischen BD und Ökosystemfunktionen zu erkennen
\begin{itemize}
	\item \textbf{“Removal Experiments”:} Arten werden aus Ökosystemen entfernt
	\item \textbf{“Synthetic Community Approach”:} Ansaat/Ansetzen neuer Modelökosysteme mit Diversitätsgradienten
	\item \textbf{“Observational Approach”:} Aufsuchen von “zufälligen” Gradienten von Diversitäts-Stufen im Freiland unter konstanten Umweltbedingungen
	\item \textbf{“Virtual Approach”:} Biodiversitätsexperimente im Computer
\end{itemize}

\textcolor{red}{Beschreibung unterschiedlicher Experimente? (Cedar Creek, Biodepth, Jena Experiment, BIOTREE, ...)}

\newpage
\subsubsection{Themen zu BD-EF Experimenten}
\begin{itemize}
	\item Produktivität und die Mechanismen positiver BD-P Zusammenhänge:
	\begin{itemize}
		\item Komplementaritätseffekt
		\item Selektionseffekt
		\item Partitionierung der Effekte
		\item Pathogeneffekte
		\item Stärke des Produktivitätseffekts
	\end{itemize}
	\item Stabilität von Ökosystemen
	\item Ausgewählte andere Ökosystemfunktionen
	\item Multifunktionalität (Konzepte und Ergebnisse)
	\item Form der Zusammenhänge
	\item Funktionelle Diversität vs. Artenreichtum
\end{itemize}

\textbf{Der Biomasse-Effekt in Jena am Anfang:} Konsistente, positive (log-)lineare Beziehung zwischen Artenzahl und Anzahl FG und Biomasseproduktion
\\
\textbf{Cedar Creek:}
\begin{itemize}
	\item Biomasse-Produktion nimmt mit zunehmender Diversität zu
	\item BD-Effekte werden mit der Zeit stärker / und sättigen nicht
\end{itemize}

\textbf{Der Faktor Zeit im Fokus:} Viele BD-EF-Zusammenhänge werden mit der Zeit stärker

\newpage
\subsubsection{BD-EF Mechanismen}
\begin{itemize}
	\item \textbf{Selektionseffekt:} Je mehr Arten in einer Mischung sind, desto wahrscheinlicher ist es eine besonders leistungsfähige Art im Artenpool zu haben
	\item \textbf{Komplementarität:} Die unterschiedliche Spezialisierung von Arten führt dazu, dass mehr Arten gemeinsam mehr Resourcen (Licht, Wasser, Nährstoffe) akquirieren können oder sich gegenseitig unterstützen
	\item Beide Effekte führen dazu, dass Mischungen mit mehr Arten produktiver/performanter sind
\end{itemize}

\underline{\textbf{Mechanismen: Selektionseffekt}}
\begin{itemize}
	\item Eine oder wenige Art(en) mit speziellen Eigenschaften (“traits”, z.B. hohe RGR) haben einen starken Effekt auf Ökosystemprozess (z.B. Produktivität).
	\item Mit zunehmender BD: höhere Wahrscheinlichkeit, dass diese Art im System vorhanden ist (d.h. “selektiert”).
	\item Falls diese Arten in Mischung dominant werden, nimmt die Produktivität mit zunehmender Diversität im Mittel zu.
	\item Interaktionen zwischen Arten: Konkurrenz, Konkurrenz-Ausschluss
\end{itemize}

\textbf{Sampling effect und Umweltfilter}
\begin{itemize}
	\item Diversität erlaubt es, Arten dabei zu haben, die unter den gegebenen Umweltbedingungen optimal funktionieren
	\item Reduzierte Diversität „greift“ das optimale Spektrum nicht ab
\end{itemize}

\textcolor{red}{Wie sieht BD-Biomasse Kurve aus... Wenn sich Arten mit unterschiedlichen Produktivitäten den Platz gleichmäßig aufteilen und nicht interagieren?}
\\\\
\textbf{Sampling effect}
\begin{itemize}
	\item Zufällige Auswahl der Arten aus einem festen Artenpool
	\item “non-transgressive overyielding”: Es gibt eine positiven BD-EF Effekt, aber keine Mischung produziert mehr als die beste Monokultur
\end{itemize}
	
\underline{\textbf{Mechanismen: Komplementarität}}
Konkurrenz und positive Interaktionen
\begin{itemize}
	\item Nichen-Differenzierung und Ressourcen- Komplementarität
	\item Gegenseitige Unterstützung: “Facilitation”.
	\item “Overyielding”: Produktion in Mischung höher als erwartet. Basis: Produktion in Monokulturen aller Arten.
	\item Interaktionen zwischen Arten: Diversität und Koexistenzfördernde Interaktionen
\end{itemize}

\textbf{Komplementarität}
\begin{itemize}
	\item Räumliche/zeitliche Heterogenität der limitierenden Faktoren
	\item Jede Art hat ihr Wuchsoptimum in einer spezifischen Region des 2D-Nischenraumes
	\item Mit jeder neu hinzukommenden Art wird der Nischenraum besser ausgenutzt
\end{itemize}

\textbf{“Overyielding”:} In Mischung produzieren Arten mehr Biomasse als in Monokultur und die Gesamtbiomasse nimmt zu\\
\textbf{„Transgressive Overyielding“:} wenn die Mischungen performanter sind als die besten Monokulturen
\\\\
\textbf{Komplementarität konkret}
\begin{itemize}
	\item \textbf{Räumliche Komplementarität}
	\begin{itemize}
		\item Architektur/Schattentoleranz $\rightarrow$ Bessere Ausnutzung des
Kronenraums
		\item Wurzeltiefe und Durchwurzlungsintensität: $\rightarrow$ Bessere Ausnutzung des Bodenraums
	\end{itemize}
	\item \textbf{Zeitliche Komplementarität}
	\begin{itemize}
		\item Phänologie $\rightarrow$ bessere Ausnutzung der Vegetationsperiode
	\end{itemize}
	\item \textbf{Physiologische Komplementarität}
	\begin{itemize}
		\item Unterschiedliche Nährstoffpräferenzen (z. B. Ammonium vs. Nitratpflanzen $\rightarrow$ bessere Ausnutzung des verfügbaren N
	\end{itemize}
	\item \textbf{Trophische (Interakions-) Komplementarität}
	\begin{itemize}
		\item Fressfeinde / Pathogene $\rightarrow$ Verdünnungseffekt
	\end{itemize}
\end{itemize}

\textbf{Nichen-Differenzierung}\\
Räumliche Heterogenität: vertikal
\begin{itemize}
	\item Unterschiede in der ober- und unterirdischen Architektur: Kronenraum-/Wurzelraum-Stratifizierung
\end{itemize}

\subsubsection{„Additive Partitioning“}
\begin{itemize}
	\item \textbf{Ziel 1:} Quantifizieren eines Netto- Diversitätseffektes
	\item \textbf{Ziel 2:} Quantifizieren der jeweiligen Beiträge von selection und complementarity zum Netto-Effekt
\end{itemize}

\textbf{Erwartete Erträge}\\
Welchen Ertrag (oder andere Funktion) würde ich basierend auf den Monokulturerträgen in den Mischungen erwarten?\\
$RY_{E,i}=\frac{Y_{E,i}}{M_i}$\\

Was beobachte ich tatsächlich?\\
$RY_{O,i}=\frac{Y_{O,i}}{M_i}$\\

mit $Y_{E,i}$=Erwarteter Ertrag nach Anteil, $Y_{O,i}$=Beobachteter Ertrag der Art i und $M_i$=Monokulturertrag der Art i

\textbf{Auf Plotebene}
\begin{itemize}
	\item Aufsummieren der Einzelbeiträge der Arten
	\item Diskrepanz auf Plotebene
	\item Anwenden der „Price“-Gleichung: $\Delta Y=\underbrace{N  \cdot  \overline{\Delta RY}\overline{M}}_{complementarity}+\underbrace{N \cdot cov(\Delta RY, M)}_{selection}$
\end{itemize}

\textbf{Erklärung}
\begin{itemize}
	\item Der Komplementaritätseffekt ist hoch, wenn die Diskrepanzen ($\Delta RY$) zwischen Erwartet und Beobachtet im Mittel positiv sind (einzelne können ja durchaus negativ sein)
	\item Der Selektionseffekt ist hoch, wenn diejenigen Arten mit hohen Biomassen in der Monokultur, die höchsten positiven Diskrepanzen haben, i.e. sie haben sich ausgebreitet
\end{itemize}

\textbf{Sampling vs. Komplementarität}
\begin{itemize}
	\item Nicht sich gegenseitig ausschliessend, sondern Übergänge.
	\item Anfangs v.a. Selection-Effekte: exponentielles Wachstum der Arten, welche sich schnell entwickeln und die Mischungen zunächst dominieren.
	\item Mit der Zeit wird Komplementarität stärker, durch Konkurrenz und positiven Interaktionen.
\end{itemize}

\textbf{Pathogeneffekte}
\begin{itemize}
	\item Es ist nicht so, dass artenreiche Mischungen „besser“ funktionieren
	\item Artenarme Mischungen können auch „schlechter“ funktionieren. Warum?
	\begin{itemize}
		\item Pathogenbefall / Akkumulation von Pathogenen
		\item Entstehung eines Herbivoriedruckes
	\end{itemize}
\end{itemize}

\subsubsection{Stabilität von Ökosystemen}
Mit abnehmender Variabilität steigt die Vorhersagbarkeit!
\\\\
\textbf{„Insurance“ bzw. „Portfolio“ Effekt}
Falls jede Art anders und unabhängig auf Umweltvariabilität reagiert, wird der Mittelwert mit zunehmender Artenzahl stabiler (weniger variabel)

\textbf{Resistenz und Resilienz}
\begin{itemize}
	\item Resistenz: Widerstandsfähigkeit gegenüber Störungen
	\item Resilienz: Fähigkeit sich nach einer Störung zu erholen, Resilienz hat keinen Zusammenhang mit Biodiversität
	\item Stabilität = Resistenz + Resilienz
\end{itemize}

\subsubsection{Ausgewählte andere Ökosystemfunktionen}
Speicherung von Bodenkohlenstoff\\
Düngung vs. Diversität: Biologische Mechanismen (= Interaktionen zwischen verschiedenen Arten) sind effektiver als Düngung!\\

Erste Meta-Analyse: BD-Effekte auf eine ganze Reihe von Ökosystemfunktionen
\\
\textbf{Gruppen von Funktionen}\\
\begin{itemize}
	\item Kohlenstoff- (d.h. auch Biomasse-) und wasserbasierte Funktionen reagieren positiv.
	\item Nährstoffbasierte reagieren negative, aber Achtung (!): Das kann auch erwünscht sein (z. B. Nitrataustrag)
\end{itemize}

\subsubsection{Multifunktionalität}
Quantifizierung von Multifunktionalität (Mittelwert-Ansatz)\\\\
\textbf{“Threshold approach”:} Zählen der Funktionen, die eine definierten Schwellenwert  überschreiten \textcolor{blue}{(Rechenbeispiel siehe Vorlesung)}\\\\
Je mehr Funktionen berücksichtigt werden, desto mehr Arten werden benötigt\\\\
\textbf{Multifunktionalität als Funktion von ...} Der Anteil der Arten, die das Ökosystemfunktionieren positiv beeinflussen steigt\\\\

\textbf{Artenvielfalt vs. FG-Vielfalt}
\begin{itemize}
	\item Hier ist funktionelle Vielfalt einfach als die Anzahl funktionell vordefinierter Arten abgeschätzt worden.
	\item Merkmale sind nur indirekt verwendet worden
\end{itemize}