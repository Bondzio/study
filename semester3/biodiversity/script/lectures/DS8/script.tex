\section{DS8}
\textbf{Merkmale in einer Gemeinschaft - Filter, Nischen, Koexistenz}
\begin{itemize}
	\item In einer Pflanzengemeinschaft koexistieren Arten mit unterschiedlichen Eigenschaften
	\item Was könnte wichtig sein?
	\begin{itemize}
		\item Mittlere Ausprägung des Merkmals
		\item Vorhandensein eines bestimmten Merkmals
		\item Das Ausmaß der Unterschiedlichkeit von Merkmalen (Gesamtspanne, Abdecken der Spanne etc.)
	\end{itemize}
\end{itemize}

\textbf{Was kann man daraus lernen? (Assembly rules)}
\begin{itemize}
	\item Wirken Filtermechanismen?
	\begin{itemize}
		\item Hohe Gleichartigkeit von Merkmalen $\rightarrow$ Starke Filter wirken
		\item Welche Merkmale sind besonders gleichartig? $\rightarrow$ Rückschlüsse auf Art des Filters
		\item Welche mittlere Ausprägung haben Merkmale $\rightarrow$ Ebenfalls Rückschlüsse auf Art des Filters
	\end{itemize}
	\item Wirken Nischenmechanismen?
	\begin{itemize}
		\item Hohe Verschiedenheit $\rightarrow$ Nutzung unterschiedlicher Nischen (Räume, Zeiten, Resourcen etc.)
	\end{itemize}
	\item Beide Mechanismen können gleichzeitig wirken, dann aber für unterschiedliche Merkmale
\end{itemize}

\textbf{Nischenkonzepte und Merkmale}\\
\underline{Umweltnische:} a region in a multi-dimensional space of environmental factors that affect the welfare of a species (Hutchinson,1957)\\\\
\underline{Merkmalsnische:} a region in a multi-dimensional trait space that affect the welfare of a species in a particular environment (Rosenzweig , 1987)
\\\\
Abiotischer Filter $\approx$ Identität\\
Biotischer Filter („begrenzte Ähnlichkeit“, „Nischenprozesse“) $\approx$ Diversität\\

\newpage
\textbf{Beispiel: Nischen im Regenwald}
\begin{itemize}
	\item Kraft et al. (2008): Von 150.000 Bäumen 1000 x zufällig Individuen ziehen und kleine virtuelle 20 x 20 m Plots herstellen $\rightarrow$ Verteilung von Merkmalen anschauen
(Spanne, Standardabweichung usw.)
	\item Dann Vergleich mit den gemessenen Verteilungen
	\begin{itemize}
		\item Verteilung breiter $\rightarrow$ Nischenprozesse (benachbarte Arten sind sich unähnlicher als eine zufällige Nachbarschaft)
		\item Ist die Verteilung enger $\rightarrow$ Umweltfilter macht Bäume ähnlicher als bei einer zufälligen Nachbarschaft
	\end{itemize}
\end{itemize}

\textbf{Das Prinzip}
\begin{itemize}
	\item SD Nullmodell = SD Beobachtet?
	\item Spanne Nullmodell = Spanne Beobachtet?
\end{itemize}

\textbf{Funktionelle Merkmale}
\begin{itemize}
	\item Wie setzen sich pflanzliche Merkmale in Pflanzenfitness und Ökosystemfunktionen um?
	\item Was ist die Biologie hinter den Indizes und den Statistiken?
\end{itemize}