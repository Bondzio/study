\section{DS14}
\subsubsection{Identität}
\begin{itemize}
	\item Das Vorhandensein (oder Fehlen) einer Art modifiziert Ökosystemprozesse
	\item Die Wirkung wird über Merkmale der Arten (effect traits) ausgeübt (direkte und indirekte Mechanismen)
	\item Detektion in natürlicher Vegetation sehr schwierig (Referenzen fehlen)
\end{itemize}

\textbf{Forschungsansätze}
\begin{itemize}
	\item Gezielte Experimente: die wenigsten können FG von echten Arteffekten trennen.
	\item Landnutzung (Pflanzungen, Ansaaten)
	\item Studie von Invasionsereignissen (oder seltener Aussterbeereignissen)
	\item Vergleich von ähnlichen Regionen mit unterschiedlicher Artenausstattung
	\item Theoretische Extinktionsszenarien
\end{itemize}

\textbf{Beobachtete Ökosystem-Effekte von Pflanzenidentität}\\
\begin{itemize}
	\item Geomorphologie
	\begin{itemize}
		\item Erosion
		\item Sedimentation
		\item Microrelief
		\item Abflussrinnen
	\end{itemize}
	\item Wasserhaushalt
	\begin{itemize}
		\item Wasserhaltekapazität
		\item Grundwassertiefe
		\item Oberflächenabfluss
	\end{itemize}
	\item Biogeochemie
	\begin{itemize}
		\item Mineralisierungsrate
		\item Immobilisierung
		\item Wasserchemismus
	\end{itemize}
	\item Störungen
	\begin{itemize}
		\item Typ
		\item Frequenz
		\item Intensität
		\item Dauer
	\end{itemize}
\end{itemize}

\textbf{Population / Biozönose}
\begin{itemize}
	\item Bestandesstruktur
	\begin{itemize}
		\item Neue Lebensformen (e.g. Baum)
		\item Vertikale Schichtung
	\end{itemize}
	\item Regeneration
	\begin{itemize}
		\item Allelopathie
		\item Mikroklima
		\item Physikalische Barriere
	\end{itemize}
	\item Konkurrenz um Resourcen
	\begin{itemize}
		\item Lichtabsorption
		\item Wasseraufnahme
		\item Nährstoffaufnahme
		\item Raumbelegung
	\end{itemize}
\end{itemize}