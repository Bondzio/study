\section{DS13}
\subsubsection{„Fundamental Traits“}
\begin{itemize}
	\item 59 species
	\item 37 plant traits (Statur, Blätter, Wurzeln)
	\item Gemessene Merkmale: Acht orthogonale Syndrome (PCA)
\end{itemize}

\textbf{Fragestellung}\\
Merkmal $\rightarrow$ Maß für Unterschiedlichkeit $\rightarrow$ Funktion\\
Merkmal $\rightarrow$ Maß für mittlere Ausprägung $\rightarrow$ Funktion\\
\\
Functional Diversity: Wie viele? (8 FD-Traits)\\
Functional Identity: Wer? (8 FI-Traits)\\
\\
\textbf{Diversität vs. Identität}\\
Identität ist generell wichtiger

\underline{\textbf{Fazit}}
\begin{itemize}
	\item Alle 8 Syndrome sind wichtig für Funktionen. Es gibt kein „Master Syndrome“.
	\item Wenn man mehrere Funktionen gleichzeitig vorhersagen will braucht man sehr viel Merkmalsinformation
	\item Ein Merkmalssyndrom ist nicht genug; Meistens werden mindestens 5 Syndrome gebraucht
	\item Identität ist wichtiger als Diversität
\end{itemize}

\textbf{Auf großer Skala?}\\
Hypothesen
\begin{itemize}
	\item Gesetz von Liebig: Produktivität wird vom am stärksten limitierenden Faktor bestimmt (also nur Umwelt)
	\item Stress gradient hypothesis: Die relative Bedeutung von Komplementarität/Facilitation steigt mit zunehmender Resourcen-Limitierung
\end{itemize}

\textbf{Fazit}
\begin{itemize}
	\item Das Gesetz von Liebig trifft zu
	\begin{itemize}
		\item Niederschlag limitiert in Spanien, Temperatur in Finnland
		\item Wenn beide ausreichend
	\end{itemize}
	\item Es gibt zusätzliche „Diversitätskontrolle“
	\begin{itemize}
		\item Komplementarität (= FD hat wichtigen Erklärungsbeitrag) ist wichtig bei Wasserstress (bestätigt stress gradient hypothesis), aber nicht bei Temperatur/Nährstoff-Limitierung (Finnland)
		\item Identität ist wichtig bei den Extremen des Klimagradienten (spezifische Anpassungen)
		\item Diversitätskontrolle is schwach unter günstigen Bedingungen
	\end{itemize}
\end{itemize}

\textbf{Kontext-Abhängigkeit von BEF-Beziehungen}\\
\textcolor{red}{Was kann man hier mitnehmen???}